\documentclass[a4paper,12pt]{article}
\usepackage{amsmath,amssymb}
\usepackage{amsfonts}
\usepackage[LGR]{fontenc}
\usepackage[greek]{babel}
\usepackage{graphicx}
\usepackage{tikz}

\begin{document}

\section*{Σημειώσεις για Εξετάσεις}

\subsection*{Περιοδικό Σήμα}
\(\forall t \ \text{και} \ \forall k \in \mathbb{N}:\)
\[
x(t)=x(t+kT)
\]
\text{όπου} \( T \) \text{είναι η περίοδος:}

Για περιοδικό σήμα διακριτού χρόνου με περίοδο \(N \in \mathbb{N}\) δειγμάτων ισχύει \( \forall n,k \in \mathbb{N}:\)
\[
x[n]=x[n+kN]
\]
Προσοχή στη διάγνωση της περιοδικότητας σημάτων διακριτού χρόνου. Π.χ. ένα ημίτονο διακριτού χρόνου είναι περιοδικό σήμα μόνο εάν η περίοδος του είναι θετικός ακέραιος αριθμός.

\subsection*{Περιττό και Άρτιο Σήμα}
Κάθε σήμα μπορεί να εκφραστεί ως άθροισμα ενός άρτιου \(x_e(t)\) και ενός περιττού \(x_o(t)\) σήματος.
\[
x(t)=x_e(t)+x_o(t)
\]
Όπου:
\[
x_e(t)=\frac{1}{2}(x(t)+x(-t)) \\
x_o(t)=\frac{1}{2}(x(t)-x(-t))
\]

\subsection*{Σήμα Ενέργειας και Σήμα Ισχύος}
Σήμα ενέργειας: \(0<\mathcal{E}_x<\infty\)\\
Σήμα ισχύος: \(0<\mathcal{P}_x<\infty\)\\
Ένα σήμα είναι είτε ενέργειας, είτε ισχύος είτε τίποτα από τα δύο.\\
Σήμα ενέργειας: Πεπερασμένη ενέργεια, μηδενική ισχύ.\\
Σήμα ισχύος: Πεπερασμένη ισχύς, άπειρη ενέργεια.\\

\subsubsection*{Ενέργεια:}
\[
\mathcal{E}_x = \underset{T \to \infty}{\lim} \int\limits_{-T}^{T}|x(t)|^2 \, dt = \int\limits_{-\infty}^{\infty} |x(t)|^2 \, dt
\]
\subsubsection*{Ισχύς (Ενέργεια ανά μονάδα χρόνου):}
\[
\mathcal{P}_x=\underset{T \to \infty}{\lim}\frac{1}{2T}\int\limits_{-T}^{T}|x(t)|^2 \, dt
\]
Όταν το \(x(t)\) περιοδικό με περίοδο \(T_0\):
\[
\mathcal{P}_x=\frac{1}{2T_0}\int\limits_{-T_0}^{T_0}|x(t)|^2 \, dt
\]

\subsection*{Μιγαδικό Σήμα}
Πολική μορφή:
\[
x(t) = Ae^{j 2\pi f_{c}t}
\]
Με \(|x(t)|=A=\sqrt{ x_{I}^2(t)+x_{Q}^2(t) }\) και \(\theta(t)=2 \pi f_{c}t=\tan^{-1}\left( \frac{x_{Q}(t)}{x_{I}(t)} \right)\)\\
Με χρήση Euler γίνεται:
\[
x(t) = x_{I}(t)+x_{Q}(t)
\]
Όπου:
\[
x_{I}(t)=A\cos(2 \pi f_{c}t) \ \text{η συμφασική συνιστώσα}
\]
\[
x_{Q}(t)=jA\sin(2 \pi f_{c}t) \ \text{η ορθογώνια συνιστώσα}
\]
Μπορούμε να πάρουμε τη συμφασική \(x_I(t)\) και την ορθογώνια \(x_Q(t)\) συνιστώσα κάνοντας χρήση του τύπου \(\cos(\alpha + \beta)=\cos(\alpha)\cos(\beta)-\sin(\alpha)\sin(\beta)\). Εφαρμόζοντας αυτό στο σήμα \(x(t)=A(t)\cos(2 \pi f_c t + \theta(t))\) παίρνουμε:
\[
x(t)=\overbrace{A(t)\cos(\theta(t))\cos(2 \pi f_c t)}^{x_I(t)}-\overbrace{A(t)\sin(\theta(t))\sin(2 \pi f_c t)}^{x_Q(t)}
\]

\subsection*{Θόρυβος Σήματος}
Ένα ενθόρυβο σήμα:
\[
y(t)=x(t)+n(t)
\]
Όπου \(n(t)\) είναι ο θόρυβος.

Η ποιότητα του συστήματος μπορεί να καθοριστεί από τον δείκτη απόδοσης συστήματος SNR (Signal to Noise Ratio):
\[
\text{SNR}= \frac{\overbrace{E[x^2(t)]}^{\text{Ισχύς σήματος}}}{\underbrace{E[n^2(t)]}_{\text{Ισχύς θορύβου}} }
\]

\subsection*{Βασικές Συναρτήσεις και Ορισμοί}

\subsubsection*{DC τιμή}
DC τιμή, \(R_s^{DC}\) τιμή ενός σήματος ορίζεται η μέση τιμή του σήματος στο χρόνο:
\[
R_s^{DC}=\underset{T \to \infty}{lim} \frac{1}{2T} \int\limits^{\infty}_{-\infty} s(t) \, dt
\]
Αν το σήμα είναι περιοδικό:
\[
R_s^{DC} = \frac{1}{2T}\int\limits^{T}_{-T} s(t) \, dt
\]

\subsubsection*{RMS τιμή}
\[
R_s^{RMS}=\sqrt{\mathcal{P}_s}
\]

\subsubsection*{Dirac}
\[
\delta(t) = 
\begin{cases}
0, \quad t \neq 0 \\
\infty, \quad t = 0
\end{cases}
\]
\[
\int\limits^{\infty}_{-\infty} \delta(t) \, dt=1
\]
\[
\delta(t)=\delta(-t)
\]
\[
\mathcal{F}\{\delta(t)\}=1
\]
\[
\mathcal{F}\{1\}=\delta(f)
\]
\[
\mathcal{F}\{e^{-j 2\pi f_0 t}\}=\delta(f-f_0)
\]

\subsection*{Χρήσιμος τύπος για χρήση σε ασκήσεις}
Μπορεί να χρησιμοποιηθεί για υπολογισμό ολοκληρώματος ισχύος/ενέργειας ή για οτιδήποτε άλλο.
\[
\cos(\alpha + \beta) + \cos(\alpha + \beta) =2\cos(\alpha)\cos(\beta)
\]

\subsection*{Συνέλιξη}
\[
y(t)=h(t) \ast x(t)=\int\limits_{-\infty}^{\infty} h(t-\tau)x(t) \, d\tau
\]
Η συνέλιξη περιγράφει το σύστημα. Σημαίνει ότι πολλαπλασιάζω δύο συναρτήσεις, που τη μία την μετατοπίζω συνέχεια και ολοκληρώνω το αποτέλεσμα. Ικανοποιεί προσεταιριστική και αντιμεταθετική ιδιότητα.

\subsection*{Συνάρτηση Ετεροσυσχέτισης}
Για σήματα ενέργειας:
\[
R_{xy}(\tau)=\int\limits^{\infty}_{-\infty}x(t)y^\ast(t-\tau) \, dt
\]
\[
R_{yx}(\tau)=\int\limits^\infty_{-\infty} y(t)x^\ast(t-\tau) \, dt
\]
Ισχύει και η ιδιότητα:
\[
R_{xy}(\tau)=x(\tau) \ast y^\ast (-\tau)
\]
Για σήματα ισχύος:
\[
R_{xy}(\tau)=\underset{T \to \infty}{\lim} \frac{1}{T} \int\limits^{\frac{T}{2}}_{-\frac{T}{2}} x(t)y^\ast(t-\tau) \, dt
\]
\[
R_{yx}(\tau)=\underset{T \to \infty}{\lim} \frac{1}{T} \int\limits^{\frac{T}{2}}_{-\frac{T}{2}} y(t)x^\ast(t-\tau) \, dt
\]
Για περιοδικά σήματα ισχύος:
\[
R_{xy}(\tau)=\frac{1}{T_0} \int\limits^{\frac{T_0}{2}}_{-\frac{T_0}{2}} x(t)y^\ast(t-\tau) \, dt
\]
\[
R_{yx}(\tau)=\frac{1}{T_0} \int\limits^{\frac{T_0}{2}}_{-\frac{T_0}{2}} y(t)x^\ast(t-\tau) \, dt
\]
Ιδιότητες:
\[
R_{xy}(\tau)=R^\ast_{yx}(-\tau)
\]
\[
R_{xy}(0)=0 \Rightarrow \int\limits^\infty_{-\infty} x(t)y^\ast(t) \, dt=0
\]
Στην περίπτωση που \(R_{xy}(0)=0\) τα σήματα \(y(t),x(t)\) είναι ορθογώνια.

\subsection*{Συνάρτηση Αυτοσυσχέτισης}
Για σήμα ενέργειας:
\[
R_{x}(\tau)=\int\limits^{\infty}_{-\infty}x(t)x^\ast(t-\tau) \, dt
\]
Η συνάρτηση αυτοσυσχέτισης χρησιμοποιείται και ως δείκτης της χρονικής μεταβολής του \(x(t)\).

Η ενέργεια \(x(t)\) προκύπτει ως:
\[
\mathcal{E}_x=R_x(0)=\int\limits^\infty_{-\infty} |x(t)|^2 \, dt
\]
Για τις τιμές του \(R_x(\tau)\) αποδεικνύεται ότι:
\[
|R_x(\tau)| \leq R_x(0)=\mathcal{E}_x
\]

\subsection*{Σειρά Fourier Περιοδικών Σημάτων}
Έστω το ακόλουθο σύνολο περιοδικών μιγαδικών σημάτων (ορθογώνια μεταξύ τους):
\[
\mathcal{G}\triangleq \{e^{j2\pi k f_0 t}\}^\infty_{k=-\infty} \, k\in \mathbb{Z}
\]
Ανάπτυξη σήματος \(x(t)\) με περίοδο \(T_0 = f_0^{-1}\) σε σειρά Fourier.
\[
x(t)=\sum_{k=-\infty}^\infty x[k]e^{j2\pi k f_0 t}
\]
\[
x[k]=\frac{1}{T_0} \int\limits^{\frac{T_0}{2}}_{-\frac{T_0}{2}} x(t)e^{-j2\pi k f_0 t} \, dt \in \mathbb{C}
\]

\subsection*{Σειρά Fourier Μη Περιοδικών Σημάτων}
\[
X(f)\triangleq \mathcal{F}\{x(t)\}=\int\limits_{-\infty}^{\infty} x(t) e^{-j2\pi ft} \, dt \in \mathbb{C}
\]
\[
x(t)=\mathcal{F}^{-1}\{X(f)\} = \int\limits_{-\infty}^\infty X(f)e^{j2 \pi ft} \, df
\]
Οι συντελεστές της σειράς Fourier \(x[k]\) συνδέονται με το μετασχηματισμό Fourier ως εξής:
\[
x[k]=\frac{1}{T_0} X \left (\frac{k}{T_0} \right)
\]

\subsection*{Θεώρημα Parseval}
Η ισχύς ενός περιοδικού σήματος \(x(t)\):
\[
\mathcal{P}_x = \sum_{k=\infty}^\infty |x[k]|^2
\]
Η ενέργεια ενός μη περιοδικού σήματος \(x(t)\):
\[
\mathcal{E}_x=\int\limits^\infty_{-\infty} |X(f)|^2 \, df \left ( = \int\limits_{-\infty}^{\infty} |x(t)|^2 \, dt\right)
\]
Ο όρος \(|X(f)|^2\) ονομάζεται φασματική πυκνότητα ενέργειας του \(x(t)\). Η φασματική πυκνότητα ενέργειας ενός σήματος ενέργειας \(x(t)\) με \(X(f)=\mathcal{F}\{x(t)\}\) προκύπτει ως:
\[
\mathcal{F}\{R_x(\tau)\}=|X(f)|^2
\]
Για ένα περιοδικό σήμα ισχύος \(x(t)\) η φασματική πυκνότητα ισχύος προκύπτει ως:
\[
S_x(f) \triangleq \mathcal{F}\{R_x(\tau)\}
\]

\subsection*{Ζωνοπερατό Σήμα}
Όταν το \(x(t)\) είναι ζωνοπερατό σήμα, το \(X(f)\) είναι συγκεντρωμένο γύρω από τη συχνότητα \(f_c\).
\[
X(f)=0, \ |f-f_c|\geq W
\]
Αν \(W \ll f_c\) ονομάζεται σήμα στενής ζώνης.


\begin{tikzpicture}
    % Draw the x-axis
    \draw[->] (-7,0) -- (7,0) node[right] {};

    % Draw the y-axis (not labeled in the image)
    \draw[->] (0,-0.5) -- (0,4) node[above] {};

    % Left side triangle
    \draw[thick] (-6,0) -- (-4,3.5) -- (-2,0);

    % Right side triangle
    \draw[thick] (2,0) -- (4,3.5) -- (6,0);

    % Labels for the x-axis
    \node at (-6, -0.5) {$-f_c - W$};
    \node at (-4, -0.5) {$-f_c$};
    \node at (-2, -0.5) {$-f_c + W$};
    \node at (2, -0.5) {$f_c - W$};
    \node at (4, -0.5) {$f_c$};
    \node at (6, -0.5) {$f_c + W$};
    \node at (0, -1) {$0$};

    % Labels for the Y-axis at the top of the triangles
    \node at (-4, 3.8) {$|X(f)|$};
    \node at (4, 3.8) {$|X(f)|$};
    
    % Draw vertical lines from the axis to the peaks (optional, if needed for the visual look)
    \draw[dashed] (-4, 0) -- (-4, 3.5);
    \draw[dashed] (4, 0) -- (4, 3.5);

\end{tikzpicture}


\subsection*{Σήμα Βασικής Ζώνης}
Ένα σήμα βασικής ζώνης \(x(t)\) έχει το φάσμα του \(X(f)\) συγκεντρωμένο γύρω από τη μηδενική συχνότητα, δηλαδή \(X(f)=0, \ |f|\geq W\).

\subsection*{Μιγαδική Περιβάλλουσα}
Κάθε φυσικά υλοποιήσιμο (πραγματικό) ζωνοπερατό σήμα μπορεί να γραφεί ως:
\[
x(t)=\Re\{g(t)e^{j2 \pi f_c t}\}
\]
Το σήμα \(g(t)\) είναι το μιγαδικό σήμα πληροφορίας βασικής ζώνης ή το ισοδύναμο χαμηλοπερατό σήμα του \(x(t)\) ή η μιγαδική περιβάλλουσα.
\[
g(t)=\overbrace{x_p(t)}^\text{Προ-περιβάλλουσα}e^{-2\pi f_c t}
\]
Έστω η IQ αναπαράσταση το \(g(t)\):
\[
g(t)=x_I(t)+jx_Q(t)
\]
Τότε προκύπτει:
\[
x(t)=\Re\{g(t)e^{j 2 \pi f_c t}\}=x_I(t)\cos(2\pi f_c t)-x_Q(t)\sin(2\pi f_c t)
\]

\subsection*{Πολική Μορφή}
\[
g(t)=V(t)e^{j\theta(t)}
\]
\[
V(t)=\sqrt{x_I^2(t)+x_Q^2(t)}
\]
\[
\theta(t)=\tan^{-1}\left(\frac{x_Q(t)}{x_I(t)}\right)
\]

\subsection*{Διαμόρφωση}
Ένα διαμορφωμένο κατά πλάτος σήμα AM έχει μιγαδική περιβάλλουσα:
\[
g(t)=V(t)=A_c+m(t)
\]
Με:
\[
x_I(t)=A_c+m(t)
\]
\[
x_Q(t)=0
\]

\subsection*{Τυχαίες Διαδικασίες ή Στοχαστικές Διαδικασίες}

\subsection*{Στασιμότητα Τυχαίων Διαδικασιών}
\subsubsection*{Αυστηρά Στάσιμη (SSS)}
Η ΤΔ \(X(t)\) είναι αυστηρά στάσιμη αν \(\forall \tau\) οι ΤΔ \(X(t)\) και \(X(t+\tau)\) έχουν την ίδια στατιστική:
\[
f_{X(t)}(\mathbf{a})=f_{X(t+\tau)}(\mathbf{a})
\]

\subsubsection*{Στάσιμη Υπό την Ευρεία Έννοια (WSS)}
Μια ΤΔ \(X(t)\) λέγεται στάσιμη υπό την ευρεία έννοια αν:
\[
m_X(t)=m_X \ \forall t
\]
\[
R_X(t_i,t_j)=R_X(t_i-t_j)=R_X(\tau)
\]
Δηλαδή, η μέση τιμή της \(X(t)\) είναι σταθερή \(\forall t\) και η ΣΑΣ της εξαρτάται μόνο από τη διαφορά των δειγματολειπτημένων χρόνων και όχι από κάθε ζεύγος χρονικών στιγμών χωριστά.

Η αυτοσυνδιακύμανση είναι επίσης συνάρτηση μόνο της χρονικής μετατόπισης \(\tau\), δηλαδή:
\[
C_X(\tau)\triangleq C_X(t,t+\tau)=R_X(\tau)-m_X^2
\]
Όταν μια ΤΔ είναι στάσιμη υπό την αυστηρή έννοια τότε είναι και στάσιμη υπό την ευρεία έννοια, το αντίστροφο δεν ισχύει.

\subsection*{Φασματική Πυκνότητα Ισχύος}
Δείχνει την κατανομή της ισχύος στις συχνότητες του σήματος. Προκύπτει από τον MF της ΣΑΣ του σήματος.

Για μία στάσιμη ΤΔ (SSS ή WSS), ορίζεται ως:
\[
S_X(f)=\int\limits_{-\infty}^{\infty} R_X(\tau) e^{-j 2 \pi f t} \, d\tau
\]
\[
R_X(\tau)=\int\limits_{-\infty}^{\infty} S_X(f) e^{j 2 \pi f t} \, df
\]

\subsection*{Φίλτρο}
Έστω ότι η WSS ΤΔ \(X(t)\) διέρχεται από ένα φίλτρο ΓΧΑ με κρουστική απόκριση \(h(t)\).

Ισχύει για τη WSS ΤΔ \(Y(t)\) της εξόδου:
\[
Y(t)=X(t) \ast h(t)=\int\limits^\infty_{-\infty}X(\tau)h(t-\tau) \, d\tau
\]

Η αυτοσυσχέτιση της εξόδου:
\[
R_Y(\tau)=E[Y(t)Y(t+\tau)]
\]
Ισχύει επίσης:
\[
R_Y(\tau)=R_X(\tau) \ast h(\tau) \ast h(-\tau)
\]

\subsection*{Τυχαίες Μεταβλητές}
\subsubsection*{Εργοδικότητα ως προς τη Μέση Τιμή}
Όταν οι χρονικές μέσες τιμές των ΤΜ είναι ίσες μεταξύ τους:
\[
E[X(t_i)]=m, \ \forall i
\]

\subsubsection*{Εργοδικότητα ως προς την Αυτοσυσχέτιση}
Όταν οι χρονικές αυτοσυσχετίσεις είναι ίσες μεταξύ τους:
\[
R_{X(t_i)}(\tau)=r, \ \forall i
\]

\subsubsection*{Σημαντικότητα}
Η εργοδικότητα είναι πολύ σημαντική ιδιότητα διότι παρατηρώντας το \(X(t_i)\) για ένα \(t_i\) προκύπτει πληροφορία για ολόκληρη την ΤΔ \(X(t)\).

\subsection*{Λευκός Θόρυβος}
\subsubsection*{ΦΠΙ Λευκού Θορύβου}
\[
S_N(f)=\frac{N_0}{2}, \ f\in(-\infty,\infty)
\]

\subsection*{Φίλτρα}
Συνήθως, πρόκειται για σύστημα που ακυρώνει κάποιες από τις συχνότητες του σήματος εισόδου του επιτρέποντας τη διέλευση κάποιων άλλων.

\subsubsection*{Βαθυπερατό Φίλτρο}
\subsubsection*{Υψιπερατό Φίλτρο}
\subsubsection*{Φίλτρο Απόρριψης Ζώνης}
\subsubsection*{Ζωνοπερατό Φίλτρο}

\subsection*{Περιοχές Συχνοτήτων στα Φίλτρα}
Η περιοχή συχνοτήτων στην οποία επιτρέπεται η διέλευση αποτελεί τη ζώνη διέλευσης (passband) του φίλτρου. Εκεί που ΔΕΝ επιτρέπεται η διέλευση ονομάζεται ζώνη απόκριψης ή ζώνη αποκοπής (stopband) του φίλτρου.

\subsection*{Μπλόκ Διάγραμμα Δέκτη}
\[
\text{LNA}=\text{Low Noise Amplifier}
\]
Το πρώτο ζωνοπερατό φίλτρο ξεφορτώνεται συχνότητες ανεπιθύμητες.

Ο πρώτος μίκτης συνδυάζει την είσοδο με το σήμα του τοπικού ταλαντωτή για να φέρουμε το σήμα στην επιθυμητή συχνότητα. Ο τοπικός ταλαντωτής χρησιμοποιείται και για συγχρονισμό.

Το στενό φάσμα καθαρίζει τη μίξη.

Ο δεύτερος ενισχυτής ενισχύει μετά το καθάρισμα.
Στην έξοδο του δεύτερου ενισχυτή: \(f_1=\frac{f_{\text{out}}}{M}\)

\subsection*{Διαμόρφωση Πλάτους (AM)}
\subsubsection*{Συμβατικό AM}
Αν \(m(t)\) είναι το σήμα πληροφορίας, τότε το διαμορφωμένο σήμα \(x(t)\) έχει τη μορφή:
\[
x(t)=(A_c+m(t))\cos(2\pi f_c t)
\]
\(A_c\) και \(f_c\) υποδηλώνουν το πλάτος και τη συχνότητα του φέροντος, το οποίο γενικώς έχει τη μορφή:
\[
c(t)=A_c \cos(2\pi f_c t)
\]

Από τον ορισμό του ζωνοπερατού σήματος \(x(t)\) και της μιγαδικής περιβάλλουσας \(g(t)\) ισχύει:
\[
x(t)=\underbrace{A_c+m(t)}_{g(t)\triangleq V(t)e^{j\theta(t)}=x_I(t)+jx_Q(t)} \cos(2\pi f_c t)
\]
Απ’όπου προκύπτει ότι:
\[
x_I(t)=A_c+m(t), \ x_Q(t)=0
\]
\[
V(t)=\sqrt{x_I^2(t)+x_Q^2(t)}=|A_c+m(t)|
\]
\[
\theta(t)=\tan^{-1} \left (\frac{x_Q(t)}{x_I(t)} \right)=0
\]
Το \(V(t)\) (\(\equiv g(t)\)) του \(x(t)\) περιέχει την επιθυμητή πληροφορία.

\subsection*{Υπερδιαμόρφωση (AM)}
Όταν ισχύει \(x_I(t)=A_c+m(t)<0\) εμφανίζεται το φαινόμενο της υπερδιαμόρφωσης, δηλαδή παραμορφώνεται η περιβάλλουσα του \(x(t)\). Στην περίπτωση αυτή, η περιβάλλουσα του \(x(t)\) δε μεταβάλλεται σύμφωνα με το \(m(t)\) (σκοπός επιθυμητής διαμόρφωσης), οδηγώντας σε μη αξιόπιστη ανίχνευση του \(m(t)\) στο δέκτη (στην περίπτωση δέκτη ασύμφωνης αποδιαμόρφωσης). Πρακτικά, όταν ισχύει \(x_I(t)=A_c+m(t)<0\), η πληροφορία μεταφέρεται ανεστραμμένη.

\[
\mu = 2
\]

\subsection*{Υποδιαμόρφωση}
\[
\mu=0.5
\]

\subsection*{Τέλεια Διαμόρφωση}
\[
\mu=1
\]

\subsection*{Δείκτης Διαμόρφωσης}
\[
\mu = \frac{|\min m(t)|}{A_c}
\]
Για να αποφευχθεί το φαινόμενο της υπερδιαμόρφωσης πρέπει:
\[
A_c+\min m(t) \geq 0
\]
Για \(\min m(t)>0\) δεν υφίσταται ενδεχόμενο υπερδιαμόρφωσης μιας και \(A_c>0\). Για \(\min m(t)<0\) θα πρέπει να ισχύει για το \(\mu\):
\[
\mu \leq 1
\]

\subsection*{Φασματικό Περιεχόμενο Σήματος AM}
Έστω το φασματικό περιεχόμενο του σήματος πληροφορίας \(m(t)\):
\(B=W\), \text{όχι} \(2W\)

Το φασματικό περιεχόμενο του διαμορφωμένου κατά AM σήματος (κάνοντας MF \(X(f)=\mathcal{F}\{(A_c+m(t))\cos(2 \pi f_c t)\})\) θα είναι:

Το εύρος ζώνης του \(x(t)\): \(f_c+W-(f_c-W)=2W\).

Η διαδικασία της διαμόρφωσης διπλασιάζει το bandwidth.

\subsection*{Παραμόρφωση}
Όταν το \(f_c<W\), οι δύο πλευρές του φάσματος \(X(f)\) επικαλύπτονται.

Για την αποφυγή αυτού του φαινομένου στην πράξη χρειάζεται \(f_c \gg W\).

\subsection*{Ημιτονοειδές Σήμα Πληροφορίας}
Έστω το σήμα πληροφορίας: \(m(t)=\alpha \cos(2\pi f_m t)\), \(f_m \ll f_c\). Το διαμορφωμένο σήμα προκύπτει: \(x(t)=(A_c + \alpha cos(2\pi f_m t))\cos(2 \pi f_c t)\). Το αρχικό εύρος ζώνης είναι \(f_m-0=W\). Το εύρος ζώνης του διαμορφωμένου είναι: \((f_c+f_m)+(f_c-f_m)=2f_m=2W\).

\subsection*{Άνω Πλευρική Ζώνη}
\[
|f|>f_c
\]

\subsection*{Κάτω Πλευρική Ζώνη}
\[
|f|<f_c
\]

\subsection*{Ισχύς Σήματος AM}
Έστω η περιβάλλουσα του \(x(t)\): \(V(t)=|A_c + m(t)|\).

Αν \(S_V(f)\) είναι η ΦΠΙ του \(V(t)\), η ΦΠΙ του \(x(t)=V(t)\cos (2\pi f_c t)\) προκύπτει ως:
\[
S_x(f)=\frac{1}{4}(S_V(f-f_c)+S_V(f+f_c))
\]
Η ισχύς του \(x(t)\) προκύπτει ως:
\[
\mathcal{P}_x = \int\limits^{\infty}_{-\infty} S_x(f) \, df = ... = \frac{\mathcal{P}_V}{2}
\]

Αν η DC τιμή του \(m(t)\) είναι μηδενική \(R_m^{\text{DC}} \triangleq 0\) τότε προκύπτει:
\[
\mathcal{P}_x=\frac{A_c^2}{2} + \frac{\mathcal{P}_m}{2}
\]
\[
\mathcal{P}_V=\mathcal{P}_{A_c}+\mathcal{P}_m
\]

\subsection*{Συντελεστής Απόδοσης Ισχύος \(\eta\)}
Έστω το σήμα πληροφορίας:
\[
m(t)=\alpha \cos(2\pi f_m t) \Rightarrow \mathcal{P}_m = \frac{\alpha^2}{2}
\]
Ο συντελεστής απόδοσης ισχύος ορίζεται ως:
\[
\eta = \frac{\mathcal{P}_m}{\mathcal{P}_V}=\frac{\mathcal{P}_m}{A_c^2 + \mathcal{P}_m}
\]
Στο συγκεκριμένο παράδειγμα μπορούμε να καταλήξουμε και στο:
\[
\eta = \frac{\mu^2}{2+\mu^2}
\]
Για \(\mu = 1\) έχουμε \(33.3\%\) απόδοση. Για \(\mu \rightarrow \infty\) έχουμε αύξηση της απόδοσης. Για \(\mu >1\) το σήμα πληροφορίας έχει μεγαλύτερη ισχύ από το σήμα του φέροντος. ΔΕΝ το θέλουμε αυτό.

\subsection*{Διαμορφωτής AM}
Τάση εισόδου του κυκλώματος:
\[
V_1(t)=m(t)+A_c \cos(2 \pi f_c t)
\]
Τάση εξόδου:
\[
V_2(t)=d_1 V_1(t)+d_2 V_1^2(t)+...
\]
Αν συνεχίσουμε τις πράξεις, στο άθροισμα θα υπάρχει η έκφραση:
\[
2d_2A_c \cos(2\pi f_c t)\left(\frac{d_1}{2d_2} + m(t) \right)
\]
Για να απομονωθεί ο όρος αυτός από το \(V_2(t)\) χρησιμοποιείται ένα ζωνοπερατό φίλτρο με κεντρική συχνότητα \(f_c\) και εύρος ζώνης \(2W\). Για να αποφύγουμε την παραμόρφωση (να απομακρυνθεί ο όρος \(d_2 m^2(t)\) που δημιουργείται στο άθροισμα), θα χρειάζεται να ισχύει \(f_c>3W\).

\subsection*{Τεχνικές Αποδιαμόρφωσης}
\subsubsection*{Σύμφωνη}
Γίνεται συγχρονισμός με τη φάση του φέροντος
\begin{itemize}
\item Ακρίβεια - κόστος κυκλωμάτων.
\item Αποδιαμορφώνει κάθε AM σήμα.
\end{itemize}

\subsubsection*{Ασύμφωνη}
Δεν χρειάζεται γνώση της φάσης
\begin{itemize}
\item Φτηνή
\item Δεν αποδιαμορφώνει τα πάντα (μόνο AM με συνολικό φέρον).
\end{itemize}

Στον ασύμφωνο αποδιαμορφωτή AM, όσο η περιβάλλουσα αυξάνεται η δίοδος άγει και ο πυκνωτής φορτίζεται.

Ο ασύμφωνος αποδιαμορφωτής:
\begin{itemize}
\item Επιτυγχάνεται με τον ανιχνευτή περιβάλλουσας.
\item Κοντά στην άνοδο της τάσης \(V_{\text{in}} \equiv x(t)\) σε μία θετική ημιπερίοδο.
\item Η θετικά πολωμένη δίοδος άγει και ο πυκνωτής φορτίζεται ως τη μέγιστη τιμή του.
\item Όταν η τάση \(V_{\text{in}}(t)\) αρχίζει να λαμβάνει τιμές κάτω από τη μέγιστη τιμή:
\item Η δίοδος πολώνεται αντίστροφα και σταματά να άγει. Αυτό συμβαίνει διότι η τάση στα άκρα είναι μεγαλύτερη από την τάση εισόδου.
\item Ο πυκνωτής εκφορτίζεται με αργό ρυθμό μέσω της αντίστασης \(R\) σύμφωνα με τη σταθερά \(\tau=RC\): \(V_{\text{out}}(t)=Ve^{-\frac{t}{\tau}} \approx V(1-\frac{t}{\tau})\)
\item Η εκφόρτωση του πυκνωτή μεταξύ των θετικών κορυφών του \(V_{\text{in}} \equiv x(t)\) οδηγεί σε ripple.
\item Αν το \(\tau\) πολύ μεγάλο: Αναξιόπιστη ανίχνευση, ο πυκνωτής αδυνατεί να παρακολουθήσει το \(V(t)\).
\item Γενικά πρέπει να ισχύει: 
\(\underbrace{W<\frac{1}{\tau}}_{\text{Ο ρυθμός εκφόρτισης να είναι μεγαλύτερος από το ρυθμό μεταβολής της περιβάλλουσας}} \ll f_c\)
\item Δεν ξεχνάμε το σκοπό: ΝΑ ΑΝΤΛΗΣΟΥΜΕ ΤΗΝ ΠΕΡΙΒΑΛΛΟΥΣΑ
\item Απομάκρυνση του \(A_c\): Ως σταθερός όρος έχει άπειρη περίοδο, άρα μηδενική συχνότητα: \(\frac{1}{\infty}=0 \iff \frac{1}{\tau} = f\).
\end{itemize}

\subsection*{Διαμόρφωση AM - Overview}
\begin{itemize}
\item Με υπερδιαμόρφωση δεν μπορεί να γίνει κάτι χρησιμοποιώντας ασύμφωνο αποδιαμορφωτή.
\item Ισχύς διαμορφωμένου: \(\mathcal{P}_x=\frac{A_c^2}{2} + \frac{\mathcal{P}_m}{2}\)
\item Baseband \(\Rightarrow\) Passband
\item \(f_c \gg W\) στον Διαμορφωτή Διόδου ΑΜ
\item Το αν θα επιλέξουμε σύμφωνη ή ασύμφωνη διαμόρφωση εξαρτάται από το αν βρίσκουμε τη φάση και πηγαίνουμε με βάση αυτή.
\item Σύμφωνη: Πολλαπλασιάζω με ένα φέρον που έχει τη σωστή φάση και ξεφορτώνομαι τις περιττές συχνότητες.
\item Ασύμφωνη: Παρακολουθώ την περιβάλλουσα.
\item Στον σύμφωνο αποδιαμορφωτή δεν έχουμε υπερδιαμόρφωση και δεν εμφανίζεται το πρόβλημα του αργού πυκνωτή. Όμως, είναι πιο περίπλοκος.
\end{itemize}

\subsection*{Παραλλαγές Διαμόρφωσης AM}
\subsubsection*{Double-Side Band with Suppressed Carrier (DSB-AM-SC)}
\[
x(t)=A_cm(t) \cos(2\pi f_c t)
\]
Το αποτέλεσμα αυτό αν περαστεί από ανιχνευτή περιβάλλουσα δεν μπορεί να αντλήσει το σήμα πληροφορίας διότι το αποτέλεσμα αυτό είναι η απόλυτη τιμή του σήματος πληροφορίας. Άρα, αν και δεν έχουμε αλλοίωση, τα αρνητικά γίνονται θετικά. Συνεπώς, χρειάζομαι και τη φάση για να προσδιορίζω αν είμαι σε ανεστραμμένο κομμάτι και πως να τα διαχειριστώ συνολικά.

Ο MF διαμορφωμένου κατά DSB-AM-SC σήματος \(x(t)\):
\[
X(f)=\frac{A_c}{2}(M(f-f_c)+M(f+f_c))
\]
Η ισχύς:
\[
\mathcal{P}_x=\frac{A_c^2}{2}\mathcal{P}_m
\]
Για το συντελεστή απόδοσης ισχύος ισχύει: \(\eta = 1\)
Το bandwidth είναι ακόμα \(2W\) (οπότε έχουμε σπατάλη).

Ο συνηθέστερος διαμορφωτής για DSB-AM-SC είναι ο ισοσταθμισμένος διαμορφωτής.

\subsubsection*{Single-Side Band - AM (SSB-AM)}
Σκοπός είναι να μειώσει το εύρος ζώνης και να μην έχουμε σπατάλη.
Άνω κλάδος:
\[
x_1(t)=A_c m(t) \cos(2 \pi f_c t)
\]
Κάτω κλάδος:
\[
x_2(t)=A_c \widehat{m(t)} \sin(2 \pi f_c t)
\]
Με:
\[
\widehat{m(t)}=\mathcal{H}\{m(t)\}=m(t) \ast h(t)
\]
Τέλος, προστίθενται τα \(x_1(t)\), \(x_2(t)\) για τη δημιουργία του σήματος LSSB-AM ή αφαιρούνται για τη δημιουργία του σήματος USSB-AM.

Αποδεικνύεται ότι:
\[
X(f)=A_c\left\{\begin{array}{ll}
M\left(f-f_c\right), & f>f_c \\
0, & f<f_c
\end{array}+A_c \begin{cases}0, & f>-f_c \\
M\left(f+f_c\right), & f<-f_c\end{cases}\right.
\]

Η ισχύς του \(x(t)\) είναι:
\[
\mathcal{P}_x=A_c^2 \mathcal{P}_m
\]

Ομοίως για διαμόρφωση LSSB-AM:
\[
X(f)=A_c\left\{\begin{array}{ll}
0, & f>f_c \\
M\left(f-f_c\right), & f<f_c
\end{array}+A_c \begin{cases}M\left(f+f_c\right), & f>-f_c \\
0, & f<-f_c\end{cases}\right.
\]

Άρα αυτό που στέλνω θα έχει bandwidth \(W\). Η αποδιαμόρφωση είναι σύμφωνη, αφού γνωρίζουμε τέλεια τη φάση. Με τη χρήση κατάλληλου χαμηλοπερατού φίλτρου απομακρύνονται τα σήματα με συχνότητα: \(2f_c\). (\(2f_c > f_c \gg \text{Συχνότητα που έχει το σήμα}\)).

\subsection*{VSB-AM}
Καταργεί ένα μέρος της πλευρικής ζώνης και όχι ολόκληρη, δε δημιουργεί πρόβλημα στις χαμηλές συχνότητες του σήματος πληροφορίας. Καλύτερη φασματική απόδοση από το DSB-AM-SC, αλλά χειρότερη από το SSB-AM.

\subsection*{Διάταξη Διαμόρφωσης VSB-AM}
Χρησιμοποιείται φίλτρο VSB όπου για \(K\) μια σταθερά:
\[
H_v(f+f_c)+H_v(f-f_c)=K
\]
Φάσμα διαμορφωμένου κατά VSB-AM σήματος:
\[
X(f)=\frac{A_c}{2}(M(f+f_c)+M(f-f_c))H_v(f)
\]

\subsection*{Διαμόρφωση PM ή FM}
Η διαμόρφωση γωνίας υπερτερεί της διαμόρφωσης AM στην ποιότητα του σήματος πληροφορίας καθώς επηρεάζεται λιγότερο από πηγές θορύβου.

Το διαμορφωμένο κατά PM ή FM σήμα ορίζεται ως:
\[
x(t)=A_c \cos(\theta(t))
\]
Με:
\[
\theta(t) \triangleq 2\pi f_c t +\phi(t), \quad \phi(t)\triangleq f(m(t))
\]
Αν μας δίνεται ο δείκτης διαμόρφωσης \(\beta\) μπορούμε να πούμε:
\[
\theta_i(t)=2\pi f_c t +\beta \sin(2\pi f_m t), \ \text{με} \ f_m \text{η συχνότητα του σήματος που θέλουμε να διαμορφώσουμε} 
\]
Στιγμιαία συχνότητα:
\[
f_i(t) \triangleq \frac{1}{2\pi} \frac{d \theta(t)}{dt} = f_c + \frac{1}{2\pi} \frac{d \phi(t)}{dt}
\]
Με (στη διαμόρφωση φάσης μόνο):
\[
\phi(t) = K_p m(t)
\]
Όπου το \(K_p\) ονομάζεται ευαισθησία φάσης με μονάδα μέτρησης \text{radians}/\text{Volt}.

Ο δείκτης διαμόρφωσης \(\beta_p\) εκφράζει τη μέγιστη μετατόπιση της φάσης στο διαμορφωμένο σήμα \(x(t)\):
\[
\beta_p \triangleq \Delta \phi_{\text{max}} = K_p \max |m(t)|
\]

Στη διαμόρφωση FM ο ρυθμός μεταβολής της φάσης μεταβάλλεται γραμμικά με το \(m(t)\) μέσω μιας σταθεράς \(K_f\) που ονομάζεται ευαισθησία συχνότητας της διαμόρφωσης με μονάδα μέτρησης το \text{Hz}/\text{Volt}.
\[
\frac{d \phi(t)}{dt}=2 \pi K_f m(t)
\]
Χρησιμοποιώντας τον ορισμό της στιγμιαίας συχνότητας προκύπτει ότι:
\[
f_i(t)=f_c + K_f m(t)
\]
Από τον ορισμό του ρυθμού μεταβολής της φάσης προκύπτει ότι:
\[
\phi (t) = 2 \pi K_f \int\limits_{-\infty}^t m(\tau) \, d\tau
\]
Αν \(W\) είναι το εύρος ζώνης του \(m(t)\), ο δείκτης διαμόρφωσης \(\beta_f\) εκφράζει τη μέγιστη μεταβολή της στιγμιαίας συχνότητας από την κεντρική συχνότητα \(f_c\), κανονικοποιημένη ως προς το \(W\):
\[
\beta_f \triangleq \frac{\Delta f_{\max}}{W}=\frac{K_f \max |m(t)|}{W}
\]

Χρησιμοποιώντας την ταυτότητα:
\[
\cos(\alpha + \beta) = \cos(\alpha)\cos(\beta) - \sin(\alpha) \sin(\beta)
\]
Το διαμορφωμένο κατά γωνία σήμα προκύπτει ως:
\[
x(t) \triangleq \underbrace{A_c \cos(\phi (t))}_{x_I(t)} \cos(2\pi f_c t) - \underbrace{A_c \sin(\phi(t))}_{x_Q(t)} \sin(2\pi f_c t)
\]

Ανάλογα με το είδος της διαμόρφωσης:
\[
\phi(t)= \begin{cases}K_p m(t), & \mathrm{PM} \\ 2 \pi K_f \int_{-\infty}^t m(\tau) d \tau, & \mathrm{FM}\end{cases}
\]

Από τις \(x_I(t)\) και \(x_Q(t)\) προκύπτει ότι:
\[
V(t) = \sqrt{x_I^2(t)+x_Q^2(t)}=A_c \quad \text{δεν εμπεριέχεται το} \ m(t)
\]
\[
\tan^{-1} \left(\frac{x_Q(t)}{x_I(t)} \right) = \tan^{-1}(\tan(\phi(t))=\phi(t))
\]

Η ισχύς του διαμορφωμένου κατά PM/FM σήματος \(x(t)\):
\[
\mathcal{P}_x^{PM} = \mathcal{P}_x^{FM} = \frac{A_c^2}{2}
\]

Αν σε διαμορφωτή PM εισαχθεί το χρονικό ολοκλήρωμα του \(m(t)\), τότε προκύπτει το διαμορφωμένο σήμα:
\[
x(t)=A_c \cos(2\pi f_c t+ K_p \int\limits_{-\infty}^t m(\tau) \, d\tau)
\]

Για \(K_p = 2\pi K_f\) προκύπτει η διαμόρφωση FM:
\[
x(t)=A_c \cos(2\pi f_c t+2\pi K_f \int\limits_{-\infty}^t m(\tau) \, d\tau)
\]

Αντιστρόφως, αν σε διαμορφωτή FM εισαχθεί η χρονική παράγωγος του \(m(t)\), τότε προκύπτει το διαμορφωμένο σήμα:
\[
x(t)=A_c \cos (2\pi f_c t +2 \pi K_f \int\limits_{-\infty}^t \frac{dm(\tau)}{d\tau} \, d\tau)
\]

Για \(K_f=\frac{K_p}{2\pi}\) προκύπτει η διαμόρφωση FM:
\[
x(t) = A_c \cos(2\pi f_c t+ K_p m(t))
\]

\subsection*{Για διαμόρφωση PM και FM, το ενεργό εύρος ζώνης ορίζεται:}
\[
B=2W(\beta + 1) \ \text{με} \ \beta = \beta_p \ \text{ή} \ \beta_f
\]
\(\beta\) ορίζεται ο δείκτης διαμόρφωσης.

Διακρίνουμε τη διαμόρφωση σε διαμόρφωση στενής (NB) και ευρείας ζώνης (WB). Εξαρτάται από το \(\beta\). Αν \(\beta \ll 1\) τότε είναι στενής ζώνης, αλλιώς ευρείας ζώνης.

\subsection*{Εύρεση WBFM σήματος}
Το διαμορφωμένο κατά WBFM σήμα γράφεται:
\[
x(t)=A_c \cos(2 \pi f_c t + \beta_f \sin(2\pi f_m t)) = \Re\{A_c e^{j(2\pi f_c t +\beta_f \sin(2\pi f_m t))}\} = \Re\{\underbrace{A_c e^{j \beta_f \sin(2\pi f_m t)}}_{=g(t)} e^{j 2 \pi f_c t}\}
\]

Η μιγαδική περιβάλλουσα \(g(t)\) είναι περιοδική συνάρτηση με περίοδο \(T_m = \frac{1}{f_m} \), συνεπώς, δύναται να αναπτυχθεί σε σειρά Fourier.

Το ανάπτυγμα σε σειρά Fourier της \(g(t)\):
\[
g(t)=\sum_{n=-\infty}^{\infty} g_n e^{jn2\pi f_m t}
\]
Με:
\[
g_n=A_c J_n(\beta_f)
\]
\[
J_n(\beta_f)=\frac{1}{2\pi} \int\limits_{-\pi}^{\pi} e^{j(\beta_f \sin(x)-nx)} \, dx
\]

Άρα η μιγαδική περιβάλλουσα εκφράζεται ως:
\[
g(t)=A_c \sum_{n=-\infty}^{\infty} J_n(\beta_f)e^{jn2\pi f_m t}
\]
Οπότε το διαμορφωμένο σήμα θα είναι:
\[
\begin{aligned}
x(t) & =A_c \Re\left\{\sum_{n=-\infty}^{\infty} J_n\left(\beta_f\right) e^{j n 2 \pi f_m t} e^{j 2 \pi f_c t}\right\} \\
& =A_c \sum_{n=-\infty}^{\infty} J_n\left(\beta_f\right) \Re\left\{e^{j 2 \pi\left(f_c+n f_m\right) t}\right\} \\
& =A_c \sum_{n=-\infty}^{\infty} J_n\left(\beta_f\right) \cos \left(2 \pi\left(f_c+n f_m\right) t\right)
\end{aligned}
\]

Το αποτέλεσμα είναι ΣΟΣ. Θα προκύψει το ίδιο αποτέλεσμα στην περίπτωση διαμόρφωσης PM με \(m(t)=\alpha \sin(2\pi f_c t)\) και \(K_p=2\pi K_f\).

Το φάσμα του διαμορφωμένου κατά WBFM σήματος είναι φαινομενικά άπειρο. Ισχύει, όμως, ότι η περιβάλλουσα του \(J_n(\beta_f)\) φθίνει για κάθε σταθερή τιμή \(n\). Για \(n>\beta_f\) η τιμή του \(J_n(\beta_f)\) είναι σχεδόν ίση με το μηδέν.

Ισχύει η ιδιότητα:
\[
J_{-n}\left(\beta_f\right)= \begin{cases}J_n\left(\beta_f\right), & n \ \text{άρτιος} \\ -J_n\left(\beta_f\right), & n \ \text{περιττός} \end{cases}
\]

Το ενεργό εύρος ζώνης διαμόρφωσης γωνίας ορίζεται:
\[
B=2 f_m(\beta+1)= \begin{cases}2 f_m\left(K_p \alpha+1\right), & \mathrm{PM} \\ 2 f_m\left(\frac{K_f \alpha}{f_m}\right)+1, & \mathrm{FM}\end{cases}
\]

Ο αριθμός των αρμονικών (δηλαδή των συνιστωσών \(A_c J_n(\beta_f)\delta(f \pm (f_c+nf_m))\) στο φάσμα του \(x(t)\) που εμφανίζονται στις θετικές του συχνότητες) που περιέχονται στο ενεργό εύρος ζώνης του \(x(t)\) προκύπτει ως:
\[
N= \begin{cases}2\left\lfloor K_p \alpha\right\rfloor+3, & \mathrm{PM} \\ 2\left\lfloor\frac{K_f \alpha}{f_m}\right\rfloor+3, & \mathrm{FM}\end{cases}
\]

\subsection*{Τεχνικές Διαμόρφωσης}
\subsubsection*{Άμεση Διαμόρφωση}
Χρησιμοποιείται ταλαντωτής (VCO) ο οποίος παράγει σήματα των οποίων η στιγμιαία συχνότητα μεταβάλλεται με τον επιθυμητό τρόπο.

Έστω ότι η είσοδος του VCO είναι το σήμα \(m(t)\). Τότε, η στιγμιαία συχνότητα της εξόδου του ταλαντωτή είναι:
\[
f_i(t)=f_c + Km(t)
\]
Όπου:
\[
f_c: \text{Ονομαστική συχνότητα εξόδου του VCO που αντιστοιχεί σε μηδενική τάση εισόδου. Πρόκειται για τη συχνότητα του φέροντος.}
\]
\[
K:\text{Σταθερά που εξαρτάται από το VCO.}
\]
Το σήμα εξόδου του VCO είναι ένα διαμορφωμένο κατά FM σήμα:
\[
x(t)=A_c \cos \left(2\pi f_c t + 2 \pi K \int\limits_{-\infty}^t m(\tau) \, d\tau \right)
\]

\subsubsection*{Έμμεση Διαμόρφωση}
Για την υλοποίηση του διαμορφωτή WBFM υλοποιείται πρώτα η διαμόρφωση NBFM η οποία έπειτα μετατρέπεται σε ευρείας ζώνης.

Ακολουθεί τα παρακάτω έξι βήματα:
\begin{enumerate}
\item Η διαμόρφωση NBFM επιτυγχάνεται χρησιμοποιώντας κύκλωμα διαμόρφωσης DSB-AM-SC:
\[
x_{NBFM}(t)=A_c \cos(2\pi f_0 t + \beta_f \sin(2\pi f_m t))
\]
Όπου \(f_0\) είναι μία αρχική συχνότητα.

\item Πολλαπλασιασμός συχνότητας με παράγοντα \(n_1\):
\[
x(t)=A_c \cos(n_1 2\pi f_0 t + n_1 \beta_f \sin(2\pi f_m t))
\]

\item Στην έξοδο του μίκτη έχει πραγματοποιηθεί ετεροδύνωση, δηλαδή συχνοτική μετακίνηση του \(x(t)\) στη νέα συχνότητα \(f_{IF}=n_1 f_0 + f_1\):
\[
x(t)=A_c \cos(2\pi f_{IF}t+n_1 \beta_f \sin(2\pi f_m t))
\]

\item Μετά γίνεται πάλι πολλαπλασιασμός συχνότητας με έναν παράγοντα \(n_2\):
\[
x(t)=A_c \cos (n_2 2\pi f_{IF} t + n_1 n_2 \beta_f \sin(2\pi f_m t))
\]

\item Το τελικά διαμορφωμένο κατά WBFM σήμα έχει συχνότητα \(f_c \triangleq n_2(n_1 f_0+ f_1)\) και δείκτη διαμόρφωσης \(\beta_{f_{WB}}\sin(2\pi f_m t)\)

\item Το ζωνοπερατό φίλτρο βοηθά την απομάκρυνση συχνοτήτων που δημιουργήθηκαν από τα μη γραμμικά στοιχεία:
\[
x(t)=A_c \cos(2\pi f_c t + \beta_{f_{WB}}\sin(2\pi f_m t))
\]
\end{enumerate}

\subsection*{Αποδιαμόρφωση}
Χρησιμοποιείται ανιχνευτής περιβάλλουσας με είσοδο \(y(t)\) που μπορεί να ανακτηθεί το αρχικό σήμα.
\[
y(t) = \frac{dx(t)}{dt}
\]
Απαραίτητη προϋπόθεση το \(A_c\) να είναι χρονικά αμετάβλητο.

\subsection*{Μετατροπή Αναλογικού σε Ψηφιακό}
Σήμα διακριτού χρόνου:
\[
m[n] \triangleq m_s(nT_s)
\]

\subsection*{Θεώρημα - Κριτήριο Nyquist}
Έστω το σήμα βασικής ζώνης \(m(t)\) για το οποίο ισχύει:
\[
M(f)=0, \ |f|>W
\]
Αν \(m[n]=m(nT_s)\) είναι τα δείγματα του \(m(t)\) τα οποία λαμβάνει με συχνότητα \(f_s=T^{-1}_s\),  τότε είναι δυνατή η ακριβής ανάκτηση του \(m(t)\) από τα δείγματα \(m(nT_s)\), αν ισχύει:
\[
T_s=\frac{1}{f_s}\leq \frac{1}{2W} \Rightarrow f_s \geq 2W
\]
Το δειγματοληπτημένο σήμα \(m_s(t)\) ορίστηκε ως:
\[
m_s(t)=m(t) \sum_{n-\infty}^{\infty} \delta(t-nT_s)
\]
είναι συνεχές!

\subsection*{Φάσμα}
Το φάσμα του \(m_s(t)\) αποτελείται από άπειρα αντίγραφα του \(M(f)\), καθένα από τα οποία είναι μετατοπισμένο κατά πολλαπλάσια του \(f_s\).

Όταν δεν πληρείται το κριτήριο του Nyquist, αυτά τα πολλαπλά αντίτυπα έχουν overlap.

\subsection*{Ανακατασκευή Αναλογικού Σήματος}
Ένας τρόπος ανακατασκευής είναι με χρήση χαμηλοπερατού φίλτρου εύρους ζώνης \(W_I\) που ικανοποιεί τη σχέση:
\[
W \leq W_I \leq f_s-W
\]

Οποιοδήποτε σήμα συνεχούς χρόνου με πεπερασμένο εύρος ζώνης δύναται να εκφραστεί ως άθροισμα άπειρων διαδοχικών συναρτήσεων \(sinc(.)\) με βάρη τα δείγματα του σήματος αυτού.

\subsection*{Δειγματοληψία Παλμού}
Στην πράξη, δεν έχω Dirac, αλλά έχω παλμούς διάρκειας \(\tau\). Καθώς \(\tau \rightarrow 0\), πηγαίνω στη θεωρητική δειγματοληψία με Dirac.

Έτσι έχουμε:
\[
p(t)=\sum_{n=-\infty}^{\infty} y_p (t-nT_s)
\]
όπου:
\[
y_p=\begin{cases}
1, & t \in [-\frac{\tau}{2}, \frac{\tau}{2}] \\
0, & \text{αλλού}
\end{cases}
\]

\subsection*{Κβάντιση}
Διαδικασία αντιστοίχισης των δειγμάτων της εξόδου του δειγματολήπτη σε ένα πεπερασμένο σύνολο επιπέδου πλάτους.

Αν τα επίπεδα πλάτους κβάντισης είναι ισαπέχοντα, τότε η κβάντιση καλείται ομοιόμορφη, αλλιώς ανομοιόμορφη.

Η κβάντιση είναι μη αντιστρέψιμη απεικόνιση \(f\) ενός σήματος (ντετερμινιστικού ή στοχαστικού) \(x(t)\), που θεωρητικά οδηγεί σε απώλεια πληροφορίας, \(y(t)=f(x(t))\).Είναι κρίσιμο η πληροφορία που χάνεται να είναι όσο το δυνατόν λιγότερο χρήσιμη για την ανάκτηση του αρχικού σήματος.

Πρέπει να ελαχιστοποιήσουμε κάποια μετρική παραμόρφωσης. Συνήθως, για ντετερμινιστικά σήματα είναι η τετραγωνική απόσταση μεταξύ αρχικού και κβαντισμένου σήματος:
\[
d(x(t),y(t))=\int\limits_{-\infty}^{\infty} |x(t)-y(t)|^2 \, dt
\]

Το αντίστοιχο μέγεθος για στοχαστικά σήματα είναι η μέση τετραγωνική παραμόρφωση:
\[
E[(X(t)-Y(t))^2]=R_{XY}(0)
\]

\subsection*{Ομοιόμορφη Κβάντιση}
Αν \(L\) ο συνολικός αριθμός σταθμών, τότε θεωρούμε:
\[
L=2^R
\]
Όπου \(R\) είναι ο αριθμός των bits ανά δείγμα στην έξοδο του ADC.

Όλα τα διακριτά επίπεδα έχουν ίδιο εύρος \(\Delta\).

Τα δύο επίπεδα στα άκρα του κβαντιστή, καλούνται επίπεδα υπερφόρτωσης. Το \(\Delta\) είναι το βήμα κβάντισης και υπολογίζεται:
\[
\Delta = \frac{V_{pp}}{2^R}
\]
όπου \(V_{pp}\) είναι η peak-to-peak τιμή του αναλογικού σήματος, δηλαδή:
\[
V_{pp} = \max - \min
\]

\subsubsection*{Mid-rise (ΣΟΣ)}
Δεν διαθέτει μηδενικό επίπεδο. Τα επίπεδα χωρίζονται σε θετικά και αρνητικά και υπάρχει ανάμεσά τους συμμετρία. Συνάρτηση μεταφοράς κβαντιστή:
\[
y=\left(n-\frac{1}{2}\right) \Delta
\]
Όπου:
\[
n=-\frac{L}{2}+1,-\frac{L}{2}+2, \dots, \frac{L}{2}-1,\frac{L}{2}
\]
δηλαδή, έχουμε \(\frac{L}{2}\) θετικά και \(\frac{L}{2}\) αρνητικά επίπεδα.

\subsection*{Θόρυβος Κβαντιστή}
Έστω η στοχαστική διαδικασία \(X\) με ΣΠΠ \(f_X(a)\), δείγματα της οποίας εισάγονται σε ομοιόμορφο κβαντιστή.

Έστω \(I_k\) το \(k\)-οστό διάστημα του ομοιόμορφου κβαντιστή mid-rise, δηλαδή αν \(x \in I_k \triangleq [(k-1)\Delta, k\Delta]\), τότε \(y=y_k \triangleq \left(k-\frac{1}{2} \right)\Delta\).

Ο θόρυβος κβάντισης για το διάστημα \(I_k\):
\[
q=x-y_k \text{ με } |q| \leq \frac{\Delta}{2}.
\]
Ο θόρυβος ακολουθεί ομοιόμορφη κατανομή:
\[
f_Q(q) = 
\begin{cases}
\frac{1}{\Delta}, & q \in \left [-\frac{\Delta}{2}, \frac{\Delta}{2} \right] \\
0, & \text{αλλού}
\end{cases}
\]

\subsection*{SQNR}
Η Μ.Τ. της τ.μ. \(Q\) του θορύβου είναι μηδέν λόγω ομοιόμορφης κατανομής σε συμμετρικό ως προς το μηδέν διάστημα.
\[
SQNR = \frac{E[x^2]}{E[Q^2]} \iff SQNR=\frac{\sigma_x^2}{\sigma_Q^2}=\frac{12 \sigma_x^2}{\Delta^2}
\]
με:
\[
\sigma_x^2 = E[x^2 - \mu^2] \ \text{ώστε}\ E[x^2] = \sigma_x^2
\]

Όσα περισσότερα bits (R) έχει ο κβαντιστής, τόσο καλύτερο SQNR πετυχαίνω.

Για να πάρω τα επίπεδα υπερφόρτωσης αντί του \(V_{pp}\), πρέπει η πιθανότητα του σήματος να πάρει τιμές έξω από αυτά να είναι μηδέν.

Θυμόμαστε:
\[
\int\limits_{-\infty}^{\infty} f_X(x) \, dx = 1
\]

\subsection*{Υπερτερόδυνος Δέκτης}
Όσο πιο μεγάλες οι συχνότητες, τόσο πιο δύσκολο να φιλτράρω, διότι το φίλτρο χρειάζεται να έχει μεγάλο εύρος ζώνης. Ο υπερτερόδυνος δημιουργεί παρεμβολές σε συχνότητες που δεν το περιμένουμε.

\subsection*{$L_0$ Φίλτρο}
Διέρχονται στην αρχή διάφορα σήματα. Επιλέγεται το επιθυμητό bandwidth με τη χρήση του φίλτρου και ενισχυτή μεταβλητής συχνότητας. Ωστόσο, δεν παρουσιάζει καλή επιλεκτικότητα.

Σημείωση για ασκήσεις: Δεν λαμβάνουμε υπόψιν τους ενισχυτές και θεωρούμε ότι τα φίλτρα είναι ιδανικά.

\subsection*{Άνω και Κάτω Μετατροπή}
Αν η \(f_{IF}\) είναι κάτω ή άνω της \(f_c\).

\subsection*{Έγχυση Υψηλής και Χαμηλής Ζώνης}
Υψηλής ζώνης: \(f_I = f_c + f_{IF}\)\\
Χαμηλής ζώνης: \(f_I = |f_c - f_{IF}|\)

\subsection*{Εικονικές Συχνότητες}
Παρεμβολές από άλλα σήματα που βρίσκονται στις λεγόμενες εικονικές συχνότητες ή συχνότητες είδωλα.
\[
f_{Im} = f_c + 2 f_{IF}
\]
Για την αποκοπή αυτών υπάρχει το \(L_0\) φίλτρο. Από αυτό το πρόβλημα μπορεί να μας γλιτώσει η επιλογή άνω ή κάτω έγχυσης.

\textbf{NOTE: Στο FM και PM το εύρος ζώνης εξαρτάται από το δείκτη διαμόρφωσης (Κανόνας Carson).}

\textbf{ΣΟΣ: 4ο κεφάλαιο}

\end{document}