\documentclass[11pt, oneside]{article}   	% use "amsart" instead of "article" for AMSLaTeX format
\usepackage{geometry}                		
\usepackage{blindtext}
\usepackage[utf8]{inputenc}
\usepackage{amsmath}
\usepackage[greek,english]{babel}
\usepackage{alphabeta}
\geometry{letterpaper}                   		% ... or a4paper or a5paper or ... 
%\geometry{landscape}                		% Activate for rotated page geometry
%\usepackage[parfill]{parskip}    		% Activate to begin paragraphs with an empty line rather than an indent
\usepackage{graphicx}				% Use pdf, png, jpg, or eps§ with pdflatex; use eps in DVI mode
% TeX will automatically convert eps --> pdf in pdflatex		
\usepackage{amssymb}
\usepackage{xcolor}
\usepackage{tcolorbox}
\usepackage{pgfplots}
\usepackage{braket}
\usepackage{float}

%------FOR DARK MODE USE-------
%\pagecolor[rgb]{0,0,0}
%\color[rgb]{1,1,1}
%------------------------------

%SetFonts

%SetFonts

\newcommand{\definition}[1]{
	\begin{tcolorbox}[colback=blue!5!white,colframe=blue!75!black,title=\textbf{Ορισμός}]
		\begin{center}
			#1
		\end{center}
	\end{tcolorbox}
}

\newcommand{\property}[1]{
	\begin{tcolorbox}[colback=red!5!white,colframe=red!75!black,title=\textbf{Ιδιότητα}]
		\begin{center}
			#1
		\end{center}
	\end{tcolorbox}
}

\newcommand{\suggestion}[1]{
	\begin{tcolorbox}[colback=green!5!white,colframe=green!75!black,title=\textbf{Πρόταση}]
		\begin{center}
			#1
		\end{center}
	\end{tcolorbox}
}

\newcommand{\note}[1]{
	\begin{tcolorbox}[colback=yellow!5!white,colframe=yellow!75!black,title=\textbf{Σημείωση}]
		\begin{center}
			#1
		\end{center}
	\end{tcolorbox}
}


\title{Στατιστική Φυσική}
\author{Κωνσταντίνος Ζουριδάκης}
\date{}							% Activate to display a given date or no date

\begin{document}
\maketitle

\tableofcontents
\newpage

\section{Νόμοι της Θερμοδυναμικής}

\definition{\textbf{Πρώτος νόμος της Θερμοδυναμικής:}\\ Το ποσό θερμότητας $(Q)$ που απορροφά ή αποβάλλει ένα θερμοδυναμικό σύστημα είναι ίσο με το αλγεβρικό άθροισμα της μεταβολής της εσωτερικής του ενέργειας και του έργου που παράγει ή δαπανά το σύστημα. \[dE = \delta Q + \delta W\] όπου $E$ είναι η εσωτερική ενέργεια και $W$ είναι το έργο.}

\note{Γράφουμε $\delta$ και όχι $d$ γιατί η μεταβολή των $Q$ και $W$ είναι τόσο μικρή που δεν αλλάζει η \textbf{συνάρτηση κατάστασης}. Η συνάρτηση κατάστασης περιέχει όλες τις μεταβλητές που περιγράφουν ένα σύστημα. Π.χ. η εσωτερική ενέργεια $E$ είναι \textbf{συνάρτηση κατάστασης}.}

Ο πρώτος νόμος της Θερμοδυναμικής πρόκειται για τον αντίστοιχο νόμος της \textcolor{red}{αρχή διατήρηση της ενέργειας} της νευτώνιας μηχανικής.

\note{\textbf{Συναρτήσεις κατάστασης που αναφέρθηκαν:}\\\begin{itemize}
		\item Εσωτερική Ενέργεια: $dE=-P \, dV + T \, dS(+ \sum_i \mu_i d \, N_i)$, $E=E(V,S,N_i)$ όπου $S$ είναι η εντροπία, $V$ είναι ο όγκος, $\mu_i$ είναι το χημικό δυναμικό του τύπου $i$ και $N_i$ είναι ο αριθμός των χημικών σωματιδίων τύπου $i$. \textbf{Σχετίζεται με ανταλλαγή θερμότητας και έργου}.
		
		\item Ελεύθερη Ενέργεια (Helmholtz): $F = E - TS = F(V,T)$, $dF=-P \, dV- S \, dT$. \textbf{Σχετίζεται με (μηχανικό και μη-μηχανικό) έργο}.
		
		\item Ελεύθερη Ενέργεια (Gibbs): $G = E + PV-TS = G(P,T)$, $dG=V \, dP - S \, dT$.\textbf{Σχετίζεται με μη-μηχανικό έργο}.
		
		\item Ενθαλπία: $H=E+PV=H(P,S)$, $dH=V \, dP+ T \, dS$. \textbf{Σχετίζεται με ανταλλαγή θερμότητας και μη-μηχανικό έργο}.
\end{itemize}}

\definition{\textbf{Δεύτερος νόμος της Θερμοδυναμικής:}\\ \begin{itemize}
		\item Για κλειστό σύστημα και αντιστρεπτή μεταβολή ισχύει $\delta Q = T /, dS$.
		\item Η εντροπία ενός κλειστού συστήματος δεν μειώνεται και λαμβάνει την μέγιστη τιμή στην κατάσταση ισορροπίας.
		\item Δύο (υπο)συστήματα σε θερμοκρασίες $T_A>T_B$ ανταλλάσουν θερμότητα $\delta Q$ και συνολικά είναι $\delta S = \frac{\delta Q}{T_B}-\frac{\delta Q}{T_A}>0$, δηλαδή η θερμότητα ρέει από το θερμό στο ψυχρό.	 
\end{itemize}}

\definition{\textbf{Τρίτος νόμος της Θερμοδυναμκής:}\\ Ένας τέλειος κρύσταλλος σε θερμοκρασία μηδέν Κέλβιν έχει εντροπία ίση με το μηδέν.}

\note{Η \textbf{εντροπία} ενός συστήματος τείνει σε μία συγκεκριμένη τιμή όταν η (απόλυτη) θερμοκρασία τείνει στο μηδέν.}

Στο μάθημα αυτό θα ασχοληθούμε κυρίως με συστήματα τα οποία βρίσκονται σε ισορροπία ή κοντά σε ισορροπία. Κοντά σε ισορροπία σημαίνει πολύ αργές μεταβολές ή αλλιώς \textbf{ψευδοσταστικές μεταβολές}.

\definition{Λέμε ότι ένα σύστημα βρίσκεται σε \textbf{κατάσταση ισορροπίας} όταν
	οι μακροσκοπικές μεταβλητές του δεν αλλάζουν με τον χρόνο.}
	
Όταν ένα σύστημα βρεθεί σε ανισορροπία, μετά από κάποιο χρονικό διάστημα θα αναγκαστεί να \textit{πλησιάσει} την κατάσταση ισορροπίας. Ο χρόνος αυτός ονομάζεται \textit{χρόνος αποκατάστασης, εξισορρόπησης ή εφησυχασμού} $\tau_i$. Π.χ. στο νερό αν ρίξουμε μία σταγόνα νερό επιπλέον θα δημιουργηθούν κυματισμοί, οι οποίοι με την πάροδο του χρόνο θα πάψουν να υπάρχουν. Να σημειωθεί ότι ο χρόνος πλήρης ισορροπίας $T$ είναι πολύ μεγαλύτερος από τον χρόνο αποκατάστασης $\tau_i$.

\section{Πιθανότητες - Στατιστική}

\definition{Για μία τυχαία μεταβλητή $X$ (μία συλλογή από γεγονότα) ορίζεται η συνάρτηση πιθανότητας:
\begin{itemize}
	\item $P(X=x_i)=f(x_i)$ για \textbf{διακριτές τιμές} $x_i$. $P(X=x_i, \Psi = y_j)=f(x_i,y_j)$ σε δύο διαστάσεις.
	\item  $P(X \in [x,x+dx])=\phi (x)\, dx$ και $P(X\in [a,b])=\int_{a}^{b} \phi (x) \, dx$ για \textbf{συνεχείς τιμές}. $P(X \in [x,x+dx]\land \Psi \in [y,y+dy])=\phi(x,y) \, dx \, dy$ σε δύο διαστάσεις.
	\end{itemize}
}
	
\definition{Για δύο ασυμβίβαστα γεγονότα $\Gamma_1, \Gamma_2$ η πιθανότητα να μετρήσουμε ή το ένα ή το άλλο είναι:\[P(\Gamma_1 \lor \Gamma_2) = P(\Gamma_1) + P(\Gamma_2)\]
Γενικότερα, για ασυμβίβαστα γεγονότα $\Gamma_1, \Gamma_2, \dots, \Gamma_i$ ισχύει: \[P(\Gamma_1 \lor \Gamma_2 \lor \dots \lor \Gamma_i) = P(\Gamma_1) + P(\Gamma_2) + \dots + P(\Gamma_i)\] Το οποίο για $i$ δυνατές τιμές μας δίνει:\[P(\Gamma_1 \lor \Gamma_2 \lor \dots \lor \Gamma_i) = P(\Gamma_1) + P(\Gamma_2) + \dots + P(\Gamma_i)=1\]}

\definition{Για δύο οποιαδήποτε γεγονότα $A,B$ η πιθανότητα το να μετρήσουμε και το ένα και το άλλο είναι: \[P(A \land B) = P(A) P(B|A)\] Όπου $P(B|A)$ είναι η υπό συνθήκη πιθανότητα να μετρηθεί η τιμή $B$ αφού έχει μετρηθεί η τιμή $A$.}

\suggestion{Αν τα $A$ και $B$ είναι στατιστικώς ανεξάρτητα τότε ισχύει: \[P(B|A)=P(B)\] και \[P(A \land B) = P(A)P(B)\]}

\definition{Ισχύει: \[P(A \lor B) = P(A)+P(B) - P(B \land A)\]}

\definition{\textbf{Μέση τιμή:}\[\braket{X} = \sum_{i=1}^n x_i f(x_i)\] ή \[\braket{X} = \int_{x_{\min}}^{x_{\max}} x \phi(x) \, dx\]}

\suggestion{Αν $X$ και $\Psi$ τυχαίες μεταβλητές, τότε ισχύει $\braket{X+\Psi} = \braket{X} + \braket{\Psi}$ και \[\braket{g(X)} = \sum^n_{i=1} g(x_i) f(x_i)\] ή \[\braket{g(X)} = \int^{x_{\max}}_{x_{\min}} g(x) \phi(x) \, dx\]}

\suggestion{Αν οι στοχαστικές μεταβλητές $X$ και $\Psi$ είναι ανεξάρτητες, τότε ισχύει $\braket{X \Psi} = \braket{X} \braket{\Psi}$}

\definition{\textbf{Διασπορά:}\[\text{Var}(X) = \braket{(X-\braket{X})^2}\]}

\definition{\textbf{Τυπική Απόκλιση:} \[\sigma_X = \sqrt{\text{Var}(X)}\] Ο τύπος της διασποράς μπορεί να γραφτεί: \[\text{Var}(X) = \sigma_X^2 = \braket{X^2}-\braket{X}^2\]}

\suggestion{Αν $X$ και $\Psi$ είναι ανεξάρτητες μεταβλητές, τότε ισχύει $\sigma^2_{X+\Psi} = \sigma_X^2 + \sigma_\Psi^2$}

\definition{\textbf{Κεντρική Ροπή Βαθμού} $r$: \[\mu_r = \braket{(X-\braket{X})^r}\]}

\definition{\textbf{Συσχέτιση (ή συνδιασπορά) δύο στοχαστικών μεταβλητών:} \[\sigma_{X\Psi} = \braket{(X-\braket{X})(\Psi - \braket{\Psi})}\]}

\note{Αν $X$ και $\Psi$ είναι στατιστικώς ανεξάρτητες μεταβλητές, τότε ισχύει $\sigma_{X\Psi}=0$}

\definition{\textbf{Συντελεστής συσχέτισης:}\[\rho = \frac{\sigma_{X\Psi}}{\sigma_X \sigma_\Psi} \ \text{με} \ -1 \leq \rho \leq 1\]}

\subsection{Διωνυμική Κατανομή}

\definition{Η πιθανότητα να έχουμε 
$x$ επιτυχίες σε 
$n$ ανεξάρτητα πειράματα. όπου το κάθε ένα έχει με πιθανότητα επιτυχίας
$p$ κάθε φορά είναι:
\[P(X=x) = \begin{pmatrix}
n \\ x
\end{pmatrix} p^x (1-p)^{n-x}\] όπου: \[\begin{pmatrix} n \\ x \end{pmatrix} = \frac{n!}{x!(n-x)!}\]}

\definition{\textbf{Μέση Τιμή:}\[\braket{X} = \sum_{\nu=1}^n \braket{S_\nu} = np\] όπου $S_\nu$ είναι μία τυχαία μεταβλητή η οποία παίρνει την τιμή 1 αν στην $\nu$-οστή ρίψη έρθει κορώνα και 0 αν έρθει γράμματα σε περίπτωση ρίψης νομίσματος.}

\definition{\textbf{Τυπική Απόκλιση:}\[\sigma_x^2 =
 \sum_{\nu=1}^n \sigma_\nu^2 =
  npq = np(1-p)\]}
  
\section{Λαγκραντζιανή Δυναμική}

\definition{H Λαγκρανζιανή $L$ δίνεται από τον τύπο:\[L=T-V\] Όπου $T = \frac 1 2 m(\dot{x}^2 + \dot{y}^2 + \dot{z}^2)$ είναι η κινητική ενέργεια και $V=V(x,y,z,t)$ η δυναμική ενέργεια.}

\note{Ισχύει: \[\dot{x}=\frac{\partial x}{\partial t}\]}

Η λαγκρανζιανή $L$ δίνεται από τον τύπο:
\[L=T-V\]
Όπου $T = \frac 1 2 m(\dot{x}^2 + \dot{y}^2 + \dot{z}^2)$ είναι η κινητική ενέργεια και $V=V(x,y,z,t)$ η δυναμική ενέργεια.

\note{$\dot{x} = \frac{\partial x}{\partial t}$}

Η λαγκρανζιανή εξίσωση είναι:
\[L(\{x_i\},\{\dot{x}_i\},t) \quad i=1,\dots,n\]

Όπου $\{x_i\} = x_1,x_2,\dots,x_n$ είναι οι καρτεσιανές συντεταγμένες του σωματιδίου, $\{\dot{x_i}\} = \dot{x_1},\dot{x_2},\dots,\dot{x_n}$ είναι η ταχύτητα στις καρτεσιανές συντεταγμένες και $t$ ο χρόνος. Αν π.χ. είμαστε σε τρεις διαστάσεις, η λαγκρανζιανή εξίσωση γίνεται:
\[L(x_1,x_2,x_3,\dot{x_1},\dot{x_2},\dot{x_3},t)\]
Το ολοκλήρωμα $S = \int_{t_1}^{t_2} L(\{x_i\},\{\dot{x}_i\},t) \, dt$ υπολογίζει την \textbf{δράση}. Λέμε ότι λαμβάνει την ελάχιστη τιμή της για την κλασική τροχιά $x_{cl}(t)$.

Το ολοκλήρωμα $S = \int_{t_1}^{t_2} L(T,V) \, dt$ επίσης μας δίνει την δράση.

Η δράση είναι ουσιαστικά μία ποσότητα που περιγράφει το μονοπάτι που ακολούθησε ένα σώμα στο χώρο και τον χρόνο.

\section{Euler-Lagrange}
Euler-Lagrange
Ένα μονοπάτι $q$ είναι στατικό (ελάχιστο) σημείο της $S$ αν και μόνο αν:
\[\frac{\partial L}{\partial x_i(t)}- \frac{d}{dt}\frac{\partial L}{\partial \dot{x}_i(t)} = 0 \quad i=1,\dots,n\]

\section{Ορμή}

Η γενικευμένη ορμή $p_i$ περιγράφεται:
\[p_i = \frac{\partial L}{\partial \dot{x}_i}\]

\section{Χαμιλτονιανή Μηχανική}
Στην Χαμιλτονιανή μηχανική η περιγραφή της κίνησης γίνεται μέσω την Χαμιλτονιανής:
\[H = H(\{x_i\},\{p_i\},t)\]
Όπου για την εναλλαγή από λαγκρανζιανή γίνεται με τον μετασχηματισμό:
\[H(\{x_i\},\{p_i\},t) = \sum_i p_i \dot{x_i} - L(x_i,\dot{x_i},t)\]
Όπου τα $\dot{x_i}$ εκφράζονται ως $\dot{x_i} = f(x_i,p_i,t)$ αντιστρέφοντας τη σχέση της γενικευμένης ορμής.

Το ισοδύναμο σύστημα 2$n$ εξισώσεων Euler-Lagrange στον Χαμιλτονιανό φορμαλισμό είναι:
\[\frac{\partial H}{\partial p_i}=\dot{x_i} \quad , \quad -\frac{\partial H}{\partial x_i} = \dot{p_i}\]
Για ένα μηχανικό σύστημα η αντίστοιχη Χαμιλτονιανή είναι: $H = T+V$.

\section{Χώρος Φάσεων}
Ο χώρος φάσεων είναι ο χώρος όλων των πιθανών καταστάσεων του συστήματος. Στις καταστάσεις του συστήματος συμπεριλαμβάνονται ο χώρος, η ταχύτητα και η ορμή του συστήματος.

Η διαγραματική αναπαράσταση γίνεται με τη χρήση των εξισώσεων:
\[\frac{\partial H}{\partial p_i}=\dot{x_i} \quad , \quad -\frac{\partial H}{\partial x_i} = \dot{p_i}\]
Όπου στον $x$ άξονα έχουμε το $\dot{x}_i$ και στον $y$ άξονα έχουμε το $\dot{p}_i$.

\section{Εξίσωση Liouville}
\[\frac{dp}{dt} = \frac{\partial p}{\partial t} + [p,H] = 0\]
Όπου $[p,H]$ ονομάζεται αγκύλη Poisson.
Η εξίσωση Lioville μας λέει ότι ο ολικός ρυθμός μεταβολής της πυκνότητας καταστάσεων στον χώρο των φάσεων είναι ταυτοτικά μηδέν.

\section{Κεντρικό Οριακό Θεώρημα}
Το Κεντρικό Οριακό Θεώρημα συνδέει την **κανονική κατανομή** με οποιαδήποτε άλλη κατανομή.
Η συνάρτηση πυκνότητα της κανονική κατανομής είναι:
\[f(x)=\frac{1}{\sigma \sqrt{2\pi}}e^{-\frac{(x-\mu)^2}{2\sigma^2}}\]
Όπου $\sigma$είναι η τυπική απόκλιση και $\mu$η μέση τιμή.

\section{Διάχυση ως διαδικασία Markov}
Γενικά, στοχαστικές διαδικασίες στις οποίες δεν υπάρχει μνήμη από προηγούμενα γεγονότα ονομάζονται αλυσίδες ή διαδικασίες Markov.
Η διάχυση δίνεται:
\[\nabla^2 n(\mathbf{r},t) = \frac 1 D \frac{\partial n}{\partial t}\]
Όπου $n(\mathbf{r},t)$ είναι η πυκνότητα στη θέση $\mathbf{r}$τη στιγμή $t$και $D=\frac{a^2}{2\tau}$ με $a$ το μήκος κάθε τυχαίου βήματος και $\tau$ ο χρόνος που χρειάζεται κάθε βήμα.

\section{Μικροκαταστάσεις και Μακροκαταστάσεις}

\begin{figure}[H]
	\centering
	\includegraphics[width=0.7\linewidth]{"Images/Screenshot 2024-10-24 at 8.38.16 PM"}
	\caption{}
	\label{fig:screenshot-2024-10-24-at-8}
\end{figure}
\begin{figure}[H]
	\centering
	\includegraphics[width=0.7\linewidth]{"Images/Screenshot 2024-10-24 at 8.38.36 PM"}
	\caption{}
	\label{fig:screenshot-2024-10-24-at-8}
\end{figure}
Όλα τα στιγμιότυπα είναι ισοπίθανα από το νόμο \textbf{ίσων πιθανοτήτων}. Οι μακροκαταστάσεις που έχουν περισσότερες μικροκαταστάσεις είναι πιο πιθανές.

\subsection{Στατιστικό Βάρος}
Ορίζουμε ως \textbf{στατιστικό βάρος} $\Omega(E,V,N,a)$ τον αριθμό των μικροκαταστάσεων που αντιστοιχεί σε μια δεδομένη μακροκατάσταση ενός θερμοδυναμικού συστήματος.
$V$ είναι ο όγκος και $N$ ο αριθμός των σωμάτων και με ενέργεια σε ένα στενό παράθυρο $[E,E +\delta E]$. Το $\delta E$ πρέπει να είναι ΤΟΣΟ μικρό ώστε να μην μεταβαίνουμε από μια μακροκατάσταση σε μια άλλη μακροκατάσταση.

Για να φτάσω σε κατάσταση ισορροπίας πρέπει να γίνουν αλληλεπιδράσεις μεταξύ μορίων, δηλαδή πρέπει να δημιουργηθούν διάφορες \textbf{μικροκαταστάσεις}.

Όταν ένα σύστημα έχει ίσες πιθανότητες να βρίσκεται σε οποιαδήποτε μικροκατάσταση λέμε ότι το σύστημα βρίσκεται σε \textbf{ισορροπία}. Ένας άλλος τρόπος να το εκφράσουμε, είναι ότι όταν η κατανομή πιθανότητας των μικροκαταστάσεων ενός συστήματος δεν αλλάζει με τον χρόνο, τότε το σύστημα βρίσκεται σε \textbf{ισορροπία}.

Στο παράδειγμα με τους μαγνήτες:
\[\Omega(E,V,N) = \begin{pmatrix}N \\ n\end{pmatrix} = \frac{N!}{n!(N-n)!}\]
Όπου $N$ ο αριθμός των σωμάτων και $n$ τα σώματα που έχουν spin παράλληλα στο πεδίο.
Για ένα σύστημα $N$ μαγνητικών διπόλων μέσα σε ένα μαγνητικό πεδίο $\mathbf{B}$ η συνολική ενέργεια του συστήματος είναι:
\[E(n)=(N-2n)\mu B\]
Με $n$ δίπολα προσανατολισμένα παράλληλα και $N-n$ αντιπαράλληλα στο πεδίο.
\begin{figure}[H]
	\centering
	\includegraphics[width=0.2\linewidth]{"Images/Screenshot 2024-10-24 at 8.40.51 PM"}
	\caption{}
	\label{fig:screenshot-2024-10-24-at-8}
\end{figure}


\section{Εντροπία}
Με την βοήθεια του στατιστικού βάρους μπορούμε να ορίσουμε και την εντροπία $S$.
\[S(E,V,N,a) = k ln\Omega(E,V,N,a)\]
Όπου $k$ είναι η σταθερά του Boltzmann.
Έστω ότι έχουμε το παρακάτω σύστημα με συνολική ενέργεια $E=E_1+E_2$.
Θεωρούμε ότι τα παρακάτω υποσυστήματα μπορούν να ανταλλάξουν μεταξύ τους ενέργεια.
Όταν το σύστημα φτάσει σε ισορροπία, η εντροπία του θα πρέπει να μεγιστοτποιηθεί, δηλαδή θα πρέπει να ισχύει:
\[\frac{\partial S_1}{\partial E_1}=\frac{\partial S_2}{\partial E_2}\]
Όσο πιο κοντά το σύστημα βρίσκεται σε κατάσταση ισορροπίας, τόσο μεγαλύτερη η τυχαιότητα.
\begin{figure}[H]
	\centering
	\includegraphics[width=0.4\linewidth]{"Images/Screenshot 2024-10-24 at 8.42.11 PM"}
	\caption{}
	\label{fig:screenshot-2024-10-24-at-8}
\end{figure}

Για το παραπάνω σύστημα, η συνολική μακροκατάσταση είναι:
\[\Omega(E,V,N) = \Omega_1(E_1,V_1,N_1) \Omega_2(E_2,V_2,N_2)\]
Από το παραπάνω καταλήγουμε στη συνολική εντροπία του συστήματος:
\[S=S_1+S_2\]
Στο παραπάνω σύστημα, θεωρώντας ότι ο συνολικός όγκος του συστήματος είναι $V_1+V_2=V=\text{σταθερό}$, αλλά οι επιμέρους όγκοι $V_1,V_2$ είναι μεταβλητοί.
Στην ισορροπία, η μεγιστοποίηση της εντροπίας στην ισορροπία μας δίνει:
\[\frac{\partial S_1}{\partial V_1} = \frac{\partial S_2}{\partial V_2}\]
Ορίζουμε λοιπόν την πίεση:
\[P_i = T_i \frac{\partial S_i}{\partial V_i}, \quad i=1,2\]
Ομοίως για $N_1+N_2=N=\text{σταθερό}$, αλλά οι αριθμοί σωματιδίων $N_1,N_2$ είναι μεταβλητοί, καταλήγουμε στην εξίσωση του χημικού δυναμικού:
\[\mu_i = \frac{\partial S_i}{\partial N_i}, \quad i=1,2\]
\section{Θερμοκρασία}
Η θερμοκρασία μπορεί να οριστεί ως:
\[\frac{1}{T_i}=\frac{\partial S_i}{\partial E_i}, \quad i=1,2\]

\section{Ατέλειες Schottky}
Οι ατέλειες Schottky αφορούν ένα στερεό από το οποίο λείπουν κάποια μόρια σε κάποιες θέσεις. Για $n$ ατέλειες Schottky με $N$ αριθμό σωματιδίων, η ενέργεια είναι $E=n\epsilon$ θέτοντας ως μηδέν την ενέργεια ενός στερεού χωρίς ατέλειες. \textbf{Ακολουθούμε την προϋπόθεση ότι οι ατέλειες ΔΕΝ αλληλεπιδρούν μεταξύ τους}.

\note{Δημιουργείται χημικός δεσμός $\rightarrow$ Χαμηλώνει η ενέργεια. \\
Σπάει χημικός δεσμός $\rightarrow$ Ανεβαίνει η ενέργεια.}
\begin{figure}[H]
	\centering
	\includegraphics[width=0.4\linewidth]{"Images/Screenshot 2024-10-25 at 9.15.11 AM"}
	\caption{}
	\label{fig:screenshot-2024-10-25-at-9}
\end{figure}

\section{Συνάρτηση Επιμερισμού}
Η συνάρτηση επιμερισμού ενός συστήματος δίνεται από τον τύπο:
\[Z = \sum_r e^{-\beta E_r}\]

\section{Κατανομή Boltzmann}

Μας δίνει την πιθανότητα ένα σύστημα να έχει ενέργεια $E_r$ αν έχει συγκεκριμένη θερμοκρασία $T$.

\[p_r = \frac{e^{-\beta E_r}}{Z}\]
\end{document}