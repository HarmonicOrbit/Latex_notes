\documentclass[11pt, oneside]{article}   	% use "amsart" instead of "article" for AMSLaTeX format
\usepackage{geometry}                		
\usepackage{blindtext}
\usepackage[utf8]{inputenc}
\usepackage{amsmath}
\usepackage[greek,english]{babel}
\usepackage{alphabeta}
\geometry{letterpaper}                   		% ... or a4paper or a5paper or ... 
%\geometry{landscape}                		% Activate for rotated page geometry
%\usepackage[parfill]{parskip}    		% Activate to begin paragraphs with an empty line rather than an indent
\usepackage{graphicx}				% Use pdf, png, jpg, or eps§ with pdflatex; use eps in DVI mode
								% TeX will automatically convert eps --> pdf in pdflatex
\usepackage{listings}		
\usepackage{amssymb}
\usepackage{xcolor}
\usepackage{tikz}


%SetFonts

%SetFonts


\title{Αρχιτεκτονική 2 Σημειώσεις για εξετάσεις}
\author{Κωνσταντίνος Ζουριδάκης}
\date{}							% Activate to display a given date or no date

\begin{document}
\maketitle

\section{Δυναμική ενέργεια και ισχύς}
\[\text{Δυναμική Ενέργεια} = \frac{1}{2} C V^2 \]
 
 \[\text{Δυναμική Ισχύς} = \frac{1}{2} C V^2 \overbrace{f_v}^\text{Συχνότητα μεταβάσεων}\]
 
\section{Κόστος ολοκληρωμένων κυκλωμάτων}

 \[ \text{Cost}= \frac{C_3+CT_3+C_{PT}}{Y}\]  
 Όπου: \\
 $\text{Cost} = \text{Κόστος ολοκληρωμένου κυκλώματος}$ \\
 $C_3 = \text{Κόστος κύβου}$ \\
 $CT_3 = \text{Κόστος δοκιμής κύβου}$ \\
 $C_{PT} = \text{Κόστος συσκευασίας και τελικής δοκιμής}$ \\
 $Y = \text{Εσοδεία τελικής δοκιμής}$
 
 \[C_3 = \frac{C_{PL}}{CPL_3 \times Y_3}\]
 Όπου:\\
 $C_{PL} = \text{Κόστος πλακιδίου}$ \\
 $CPL_3 = \text{Κύβοι ανά πλακίδιο}$ \\
 $Y_3 = \text{Εσοδεία κύβων}$
 
 \[CPL_3 = \frac{\pi \times \frac{(d_{PL})}{2}^2}{S_3} - \frac{\pi \times d_{PL}}{\sqrt{2 \times S_3}}\]
 Όπου; \\
 $d_{PL} = \text{Διάμετρος πλακιδίου}$ \\
 $S_3 = \text{Επιφάνεια κύβου}$ \\

\textbf{Bose-Einstein τύπος:}
\[Y_3 = Y_{PL} \frac{1}{(1 + \text{imp} \times S_3)^N}\]
Όπου:\\
$Y_{PL} = \text{Εσοδεία πλακιδίων}$ \\
$\text{imp} = \text{Ατέλειες ανά μονάδα επιφανείας}$\\
$N = \text{Συντελεστής πολυπλοκότητας διαδικασίας}$

\section{Φερεγγυότητα}

\[MTBF = MTTF + MTTR\]

\[\text{Availability} = \frac{MTTF}{MTBF}\]
Όπου: \\
$MTTF = \text{Mean time to failure}$\\
$MTTR = \text{Mean time to repair}$\\
$MTBF = \text{Mean time between failures}$

\section{Μέτρηση της απόδοσης}

Επιτάχυνση του $X$ σε σχέση με τον $Y$:\\

\[\text{Speedup}_{XY} = \frac{\text{Χρόνος Εκτέλεσης}_Y}{\text{Χρόνος Εκτέλεσης}_X}\]

\section{Νόμος του Amdahl}

\[\text{Χρόνος Εκτέλεσης}_{\text{νέος}} = \text{Χρόνος εκτέλεσης}_{\text{παλιός}} \times \left((1-\text{Κλάσμα}_{\text{βελτίωσης}}) + \frac{\text{Κλάσμα}_{\text{βελτίωσης}}}{\text{Επιτάχυνση}_{\text{βελτίωσης}}}\right)\]

\[\text{Επιτάχυνση}_{\text{συνολική}} = \frac{\text{Χρόνος εκτέλεσης}_{\text{παλιός}}}{\text{Χρόνος εκτέλεσης}_{\text{νέος}}} = \frac{1}{(1-\text{Κλάσμα}_{\text{βελτίωσης}})+\frac{\text{Κλάσμα}_{\text{βελτίωσης}}}{\text{Επιτάχυνση}_{\text{βελτίωσης}}}}\]

\section{Εξίσωση απόδοσης επεξεργαστή}

\[\text{Χρόνος CPU} = \text{Κύκλοι ρολογιού CPU για ένα πρόγραμμα} \times \text{Χρόνος κύκλου 	ρολογιού}\]

\[ CPI = \frac{\text{Κύκλοι ρολογιού CPU για ένα πρόγραμμα}}{\text{Πλήθος εντολών}}\]

\[\text{Χρόνος CPU} = \text{Πλήθος εντολών} \times \text{Κύκλοι ανά εντολή} \times \text{Χρόνος κύκλου ρολογιού}\]

\[\frac{\text{Εντολές}}{\text{Πρόγραμμα}} \times \frac{\text{Κύκλοι ρολογιού}}{\text{Εντολή}} \times \frac{\text{Δευτερόλεπτα}}{\text{Χρόνος κύκλου}} =
 \frac{\text{Δευτερόλεπτα}}{\text{Πρόγραμμα}} = \text{Χρόνος CPU}\]

\[\text{Κύκλοι ρολογιού CPU} = \sum_{i=1}^n \text{Πλήθος εντολών}_i \times CPI_i\]

\[\text{Χρόνος CPU} = \left(\sum_{i=1}^n \text{Πλήθος εντολών}_i \times CPI_i \right) \times \text{Χρόνος κύκλου ρολογιού}\]

\section{Αστοχίες}

\[\frac{\text{Αστοχίες}}{\text{Εντολή}} = \frac{\text{Ρυθμός αστοχίας} \times \text{Προσπελάσεις μνήμης}}{\text{Πλήθος εντολών}} = \text{Ρυθμός αστοχίας} \times \frac{\text{Προσπελάσεις μνήμης}}{\text{Εντολή}} \]

\[\text{Μέσος χρόνος προσπέλασης μνήμης} = \text{Χρόνος ευστοχίας} + \text{Ρυθμός αστοχίας} \times \text{Ποινή αστοχίας}\]

\section{Διοχέτευση}

\[CPI \ \text{διοχέτευσης} = CPI_I  + \mathcal{S} + \mathcal{D} + \mathcal{C}\]
Όπου: \\
$CPI_I = CPI \ \text{ιδανικής διοχέτευσης}$ \\
$\mathcal{S} = \text{Καθυστερήσεις δομής}$ \\
$\mathcal{D} = \text{Καθυστερήσεις κινδύνων δεδομένων}$ \\
$\mathcal{C} = \text{Καθυστερήσεις ελέγχου}$

\subsection{Κίνδυνοι δεδομένων}
\begin{itemize}
	\item Ανάγνωση μετά την εγγραφή.
	\item Εγγραφή μετά την εγγραφή.
	\item Εγγραφή μετά την ανάγωνση.
\end{itemize}

Μια εντολή εξαρτάται από την ολοκλήρωση μιας
προσπέλασης δεδομένων μιας προηγούμενης
εντολής

\begin{lstlisting}
	add $s0, $t0, $t1
	sub $t2, $s0, $t3
\end{lstlisting}

\begin{figure}[h!]
	\centering
	\includegraphics[width=0.7\linewidth]{"Computer Architecture 2/screenshot001"}
	\caption{Κίνδυνος δεδομένων.}
	\label{fig:screenshot001}
\end{figure}

Το πρόβλημα μπορεί να αντιμετωπιστεί με \textbf{προώθηση}:\\
\begin{figure}[h!]
	\centering
	\includegraphics[width=0.7\linewidth]{"Computer Architecture 2/screenshot002"}
	\caption{Προώθηση}
	\label{fig:screenshot002}
\end{figure}

Το πρόβλημα \textbf{Φόρτωσης/Χρήσης} μπορεί να λυθεί επίσης:\\

\begin{figure}[h!]
	\centering
	\includegraphics[width=0.7\linewidth]{"Computer Architecture 2/screenshot003"}
	\caption{Φόρτωση/Χρήση}
	\label{fig:screenshot003}
\end{figure}

\newpage
Μπορούμε να αποφύγουμε τις καθυστερήσεις με κατάλληλο \textbf{χρονοπρογραμματισμό}:\\

\begin{figure}[h!]
	\centering
	\includegraphics[width=0.7\linewidth]{"Computer Architecture 2/screenshot004"}
	\caption{Χρονοπρογραμματισμός}
	\label{fig:screenshot004}
\end{figure}

\subsection{Κίνδυνοι δομής}

Συμβαίνει στη διοχέτευση MIPS με μία μοναδική μνήμη. Οι εντολές load/store πραγματοποιούν προσπέλαση μνήμης και απαιτείται καθυστέρηση σε περίπτωση εκτέλεσης της μίας μετά την άλλη. Αυτό μπορεί να αποφευχθεί με τη χρήση ξεχωριστών μνημών.

\newpage

\subsection{Κίνδυνοι ελέγχου}

Η προσκόμιση της επόμενης εντολής εξαρτάται από το αποτέλεσμα της διακλάδωσης.
Η διοχέτευση δεν μπορεί να προσκομίσει πάντα τη σωστή εντολή, καθώς ακόμη δουλεύει στο στάδιο ID.

Ένας τρόπος αντιμετώπισης είναι το \textbf{stall on branch}.

\begin{figure}[h!]
	\centering
	\includegraphics[width=0.7\linewidth]{"Computer Architecture 2/screenshot005"}
	\caption{}
	\label{fig:screenshot005}
\end{figure}

Ένας άλλος τρόπος αντιμετώπισης είναι με \textbf{πρόβλεψη διακλάδωσης}.

\begin{figure}[h!]
	\centering
	\includegraphics[width=0.7\linewidth]{"Computer Architecture 2/screenshot006"}
	\caption{}
	\label{fig:screenshot006}
\end{figure}

\begin{figure}[h!]
	\centering
	\includegraphics[width=0.7\linewidth]{"Computer Architecture 2/screenshot007"}
	\caption{}
	\label{fig:screenshot007}
\end{figure}

\begin{figure}[h!]
	\centering
	\includegraphics[width=0.7\linewidth]{"Computer Architecture 2/screenshot008"}
	\caption{}
	\label{fig:screenshot008}
\end{figure}

\newpage


\subsection{Επιτάχυνση λόγω διοχέτευσης}

Με την προϋπόθεση ότι όλα τα στάδια διαρκούν τον ίδιο χρόνο:

\[\text{Χρόνος μεταξύ εντολών}_\text{Με διοχέτευση} = \frac{\text{Χρόνος μεταξύ εντολών}_\text{Χωρίς διοχέτευση}}{\text{Αριθμός σταδίων}}\]

\subsection{Στάδια διοχέτευσης}

\begin{figure}[h!]
	\centering
	\includegraphics[width=0.7\linewidth]{"../../../../../../../../var/folders/tq/s8yk3mz91sb3rqmpy8996brw0000gn/T/TemporaryItems/NSIRD_screencaptureui_XAH157/Screenshot 2024-09-16 at 7.28.46 AM"}
	\caption{}
	\label{fig:screenshot-2024-09-16-at-7}
\end{figure}

\begin{figure}[h!]
	\centering
	\includegraphics[width=0.7\linewidth]{"../../../../../../../../var/folders/tq/s8yk3mz91sb3rqmpy8996brw0000gn/T/TemporaryItems/NSIRD_screencaptureui_xNBZKs/Screenshot 2024-09-16 at 7.30.18 AM"}
	\caption{}
	\label{fig:screenshot-2024-09-16-at-71}
\end{figure}

\end{document}  