\documentclass[11pt, oneside]{article}   	% use "amsart" instead of "article" for AMSLaTeX format
\usepackage{geometry}                		
\usepackage{blindtext}
\usepackage[utf8]{inputenc}
\usepackage{amsmath}
\usepackage[greek,english]{babel}
\usepackage{alphabeta}
\geometry{letterpaper}                   		% ... or a4paper or a5paper or ... 
%\geometry{landscape}                		% Activate for rotated page geometry
%\usepackage[parfill]{parskip}    		% Activate to begin paragraphs with an empty line rather than an indent
\usepackage{graphicx}				% Use pdf, png, jpg, or eps§ with pdflatex; use eps in DVI mode
								% TeX will automatically convert eps --> pdf in pdflatex		
\usepackage{amssymb}
\usepackage{xcolor}
\usepackage{tcolorbox}
\usepackage{pgfplots}
\usepackage{braket}
\usepackage{float}
\usepackage{circuitikz}
\usetikzlibrary{arrows.meta, decorations.markings}
\usepackage{mathrsfs}
\usetikzlibrary{arrows}

%------FOR DARK MODE USE-------
%\pagecolor[rgb]{0,0,0}
%\color[rgb]{1,1,1}
%------------------------------

%SetFonts

%SetFonts

\newcommand{\definition}[1]{
	\begin{tcolorbox}[colback=blue!5!white,colframe=blue!75!black,title=\textbf{Ορισμός}]
		\begin{center}
			#1
		\end{center}
	\end{tcolorbox}
}

\newcommand{\property}[1]{
	\begin{tcolorbox}[colback=red!5!white,colframe=red!75!black,title=\textbf{Ιδιότητα}]
		\begin{center}
			#1
		\end{center}
	\end{tcolorbox}
}

\newcommand{\suggestion}[1]{
	\begin{tcolorbox}[colback=green!5!white,colframe=green!75!black,title=\textbf{Πρόταση}]
		\begin{center}
			#1
		\end{center}
	\end{tcolorbox}
}

\newcommand{\note}[1]{
	\begin{tcolorbox}[colback=yellow!5!white,colframe=yellow!75!black,title=\textbf{Σημείωση}]
		\begin{center}
			#1
		\end{center}
	\end{tcolorbox}
}

\title{Κβαντική Θεωρία της Ύλης}
\author{Κωνσταντίνος Ζουριδάκης}
\date{}							% Activate to display a given date or no date

\begin{document}
\maketitle

\tableofcontents
\newpage

\section{Διαδικαστικά}

Θα έχουμε δύο δίωρα τέστ που θα πιάνουν σύνολο 6 μονάδες.
Το τελικό διαγώνισμα θα πιάνει 4 μονάδες.

Συνολικός βαθμός:

\begin{align*}
	\max\{60\% \ \text{τέστ} \ \times \ 40\% \ \text{διαγώνισμα}, 100\% \ \text{Διαγώνισμα}\}
\end{align*}

\section{Συγγράμματα}

\begin{itemize}
	\item Κβαντομηχανική 2 Τραχανάς 
	\item Principles of Quantum Mechanics Shankar
	\item Introductory Quantum Mechanics Liboff
	\item Atomic and electronic structures of solids Kaxiras
	\item Φυσική Στερεάς Κατάστασης Ι-ΙΙ Οικονόμου
\end{itemize}

\section{Βασικά Κλασικής Μηχανικής}

\subsection{Λαγκραζιανός Φορμαλισμός}

Υπολογισμός κλασικής τροχιάς (Αρχή Ελάχιστης Δράσης): \[x_{cl}(t_i + \Delta t)=\textcolor{red}{x(t_i)} + \textcolor{red}{\dot{x}(t_i)}\Delta t\]

Τα $x(t_i)$ και $\dot{x}(t_i)$ είναι δεδομένα.

Ο δεύτερος νόμος του Νεύτωνα μας λέει:

\[m_j \frac{d^2 x_j}{dt^2}=-\frac{\partial V}{\partial x_j}\]

Όπου $m_j$ είναι η μάζα του σωματιδίου $j$ και $x_j$ οι συντεταγμένες του σωματιδίου $j$.

Για τον υπολογισμό κλασικής τροχίας θα θεωρούμε ότι μεταβαίνουμε από $(x_i,t_i)$ σε $(x_f,t_f)$.

Μία global λύση για το πως καταλήγουμε από $(x_i,t_i) \rightarrow (x_f,t_f)$ είναι η Λαγκρανζιανή $L$ (μονάδα μέτρησης Joules (J)).

Ορίζουμε:\[L = T - V= L(x, \dot{x},t)\]

Όπου $T$ είναι η κινητική ενέργεια και $V$ είναι η δυναμική ενέργεια. Συνήθως τις $x$ και $\dot{x}$ τις γράφουμε σαν $q$ και $\dot{q}$ αντίστοιχα που αναπαριστούν γενικευμένες συντεταγμένες, δηλαδή μπορούν να έχουν οποιαδήποτε μορφή χωρίς απαραίτητα να είναι καρτεσιανές.

Η δράση υπολογίζεται:
\[\textcolor{blue}{S[x(t)]}=\int_{t_i}^{t_f} L(x,\dot{x}) \, dt\]

Η $S[x(t)]$ (μονάδα μέτρησης Joule-seconds (Js)) είναι Functional (function of a function).
Η κλασική τροχιά $x_{cl}$ είναι αυτή για την οποία η $S$ είναι μικρότερη.

\begin{figure}[H]
	\centering
	\includegraphics[width=0.7\linewidth]{"Images/Screenshot 2024-10-28 at 8.35.19 PM"}
	\caption{}
	\label{fig:screenshot-2024-10-28-at-8}
\end{figure}

\note{\textbf{Function vs Functional}\\ 
	\begin{center}
		\begin{figure}[H]
			\centering \textbf{Function:}
			\[
			f : x \in X \ \text{Number Field}\mapsto f(x) \in Y \ \text{Number Field}
			\]
			
			\textbf{Functional:}
			\[
			\mathscr{F} : f(x) \in Y^X \ \text{Function Space} \mapsto \mathscr{F}[f(x)] \in Z \ \text{Number Field}
			\]
			
			\[X, Y, Z \in \{\mathbb{R}, \mathbb{C}\}\]
		\end{figure}
	\end{center}
}

\subsection{Euler-Lagrange}
\[\frac{\partial L}{\partial q} - \frac{d}{dt}\frac{\partial L}{\partial \dot{q}} = 0\]

Η εξίσωση αυτή μας βοηθάει να βρούμε την εξίσωση $q(t)$ για την οποία η συναρτησιακή εξίσωση της δράσεις \textbf{εξτρεμίζεται}, δηλαδή έχει στάσιμο σημείο το οποίο είναι minimum, maximum ή saddle point.

\section{Βασικά}

Σημείωση: $\dot{x}$ είναι η χρονική παράγωγος, δηλαδή $\frac{\partial x}{\partial t}$.\\~\\

Στην κβαντομηχανική η φυσική κατάσταση ενός σώματος περιγράφεται από μια κυματοσυνάρτηση $\Psi(x_i,t)$.

Τα φυσικά μεγέθη σχετίζονται με τελεστές, δηλαδή μετασχηματισμούς που επενεργούν πάνω στις κυματοσυναρτήσεις. Π.χ. ο τελεστής θέσης $x$ που πολλαπλασιάζει την $\Psi$ με $x$.

\property{\textbf{Γραμμικοί τελεστές:}
	\[A[\psi_1(x) + \psi_2(x)]=A\psi_1(x)+A\psi_2(x), \forall \psi_1,\psi_2\]}
\section{Μεταθέτης}

Αν $A$ και $B$ δύο τελεστές, ο ορισμός του μεταθέτη λέει:

\definition{$[A,B] = AB-BA$}

Για τυχούσα κυματοσυνάρτηση $\psi$ δεν ισχύει απαραίτητα ότι $A(B\psi)=B(A\psi)$.
Ισοδύναμα, δεν ισχύει εν γένει ότι $AB=BA$, δηλαδή \textbf{οι κβαντομηχανικοί τελεστές δεν μετατίθενται απαραίτητα}.

\property{
	\textbf{Βασικές ιδιότητες μεταθετών:}
	\begin{align*}
		[A,B]&=-[B,A]\\
		[A,B+C]&=[A,B]+[A,C]\\
		[A,BC]&=[A,B]C+B[A,C]\\
		[Α,Β]^\dag &= [B^\dag,A^\dag]
	\end{align*}
	}
	
\suggestion{\textbf{Παράδειγμα μεταθέτη:}\[[x,p]=i\hbar\]}

\suggestion{\textbf{Αν δύο τελεστές μετατίθονται, τότε ισχύει η ιδιότητα:}
\[[A,B]=0\]}

\section{Η εξίσωση του Schroedinger}

Η Χαμιλτονιανή $H$ ενός συστήματος είναι ένας τελεστής που αντιστοιχεί στην συνολική ενέργεια του συστήματος (συμπεριλαμβάνεται δυναμική και κινητική ενέργεια).Ο τελεστής που αντιστοιχεί στο σύστημα συμβολίζεται $\mathbf{H}$.

\suggestion{\textbf{Με τα παραπάνω δεδομένα, ισχύει ότι:} \[\mathbf{H}\psi = i \hbar \frac{\partial \psi}{\partial t}\]}

Για $\mathbf{H}=\frac{p^2}{2m}+V(\mathbf{r})$ και με αντικατάσταση $\mathbf{p} \rightarrow-i\hbar \nabla$ προκύπτει:
\[\mathbf{H}=-\frac{\hbar^2}{2m} \nabla^2 + V(\mathbf{r})\]

Κάνοντας αντικατάσταση στην εξίσωση $\mathbf{H}\psi = i \hbar \frac{\partial \psi}{\partial t}$ καταλήγουμε:

\definition{\textbf{Η κυματοσυνάρτηση που μας δίνει την χρονική εξέλιξη του συστήματος:}\[-\frac{\hbar^2}{2m}\nabla^2\psi + V(\mathbf{r})\psi = i\hbar \frac{\partial\psi}{\partial t}\]}

\section{Χρησιμότητα Κυματοσυνάρτησης}

Το τετράγωνο της απόλυτης τιμής της $\psi(\mathbf{r})$ μας δίνει την πυκνότητα πιθανότητας.

\suggestion{Σε 1 διάσταση, για $P(x)=|\psi(x)|^2$ το γινόμενο $P(x)dx$ μας δίνει την πιθανότητα να βρούμε το σωματίδιο μεταξύ $x$ και $x+dx$.}

\definition{\textbf{Η πιθανότητα να βρούμε το σωματίδιο μεταξύ $a$ και $b$ δίνεται από το:} \[P(a\leq x\leq b) = \int_{a}^{b} |\psi(x)|^2 \, dx\]}

Μία κανονικοποιημένη κυματοσυνάρτηση μας λέει ότι αν ψάξουμε παντού, το σωματίδιο θα βρίσκεται κάπου με πιθανότητα 100\%, οπότε:

\definition{\textbf{Η πιθανότητα να βρίσκεται το σωματίδιο οπουδήποτε:} \[\int_{-\infty}^{+\infty}|\psi(x)|^2 \, dx = 1\]}

\definition{\textbf{Για δύο κυματοσυναρτήσεις $\psi(x)$ και $\phi(x)$ ορίζουμε το εσωτερικό γινόμενο ως το ολοκλήρωμα:}\[(\psi,\phi) = \int_{-\infty}^{+\infty} \psi^*(x) \phi(x) \, dx\] και ισχύει $(\psi,\phi) = (\phi,\psi)^*$}

\note{\textbf{Μιγαδικός συζυγής:}\[\text{αν} \ z=x+iy \, (x,y \in \Re), \text{τότε} \ z^* \equiv x-iy\]}

\definition{Για μία κυματοσυνάρτηση $\psi(x)$ και έναν τελεστή $A$ ορίζουμε ως \textbf{μέση τιμή} το ολοκλήρωμα:\[\braket{A}=\int_{-\infty}^{+\infty} \psi^*(x) A \psi(x) \, dx\]
Η μέση τιμή $\braket{A}$ είναι και η \textbf{αναμενόμενη τιμή} για το φυσικό μέγεθος $A$ όταν γίνεται μέτρησή του ενώ το σύστημα είναι στην κατάσταση $\psi(x)$.}

\definition{\textbf{Αβεβαιότητα μέτρησης}:\[\Delta A = \sqrt{\braket{A^2}-\braket{A}^2}\] όπου: \[\braket{A^2} = \int_{-\infty}^{+\infty} \psi^*(x) A^2 \psi(x) \, dx\] και $\braket{A}$ η μέση τιμή.}

Θέλουμε να εγγυηθούμε ότι η μέση τιμή $\braket{A}$ είναι πραγματικός αριθμός. Η ιδιότητα αυτή εξασφαλίζεται με την απαίτηση ότι τα φυσικά μεγέθη παριστάνονται στην Κβαντομηχανική με \textbf{εριμιτιανούς τελεστές}. Γενικότερα, όλοι οι τελεστές που περιγράφουν φυσικά μεγέθη \textbf{πρέπει} να είναι εριμιτιανοί.

\definition{\textbf{Ένας τελεστής $A$ είναι εριμιτιανός αν ισχύει:} \[\int \psi^* (A\phi) \, dr = \int(A\psi)^* \phi \, dr\] ή αλλιώς $(\psi,A\phi) = (A\psi,\phi)$ όπου $\psi,\phi$ κυματοσυναρτήσεις.}

Οι ιδιοτιμές ενός ερμιτιανού τελεστή είναι πραγματικοί αριθμοί και οι ιδιοσυναρτήσεις ερμιτιανού τελεστη που αντισοιχούν σε διαφορετικές ιδιοτιμές είναι ορθογώνιες μεταξύ τους.

\suggestion{\textbf{Αφού οι εριμιτιανοί τελεστές είναι πραγματικοί αριθμοί, για την μέση τιμή εριμιτιανού τελεστή ισχύει:} \[\braket{A}^* = \int[\psi^* (A\psi)]^* \, dr = \int(A\psi)^* \psi \, dr = \int \psi^* (A\psi) \, dr = \braket{A}\]
Δηλαδή, οι \textbf{μέσες τιμές} εριμιτιανών τελεστών είναι \textbf{πραγματικοί αριθμοί}.}

\definition{\textbf{Ο συζυγής ενός τελεστή $A$ είναι ο τελεστής $A^\dagger$ για τον οποίο ισχύει:}\[(\psi,A\phi)=(A^\dagger \psi,\phi)\]
Για τους \textbf{εριμιτιανούς} τελεστές ισχύει $A=A^\dagger$. Με αφορμή αυτό, οι εριμιτιανοί τελεστές ονομάζονται \textbf{αυτοσυζυγείς}.}

\property{\textbf{Ιδιότητες συζυγών τελεστών:}
\begin{align*}
	(A^\dagger)^\dagger &= A \\
	(A+B)^\dagger &= A^\dagger + B^\dagger \\
	(AB)^\dagger &= B^\dagger A^\dagger \\
	(ABC)^\dagger &= C^\dagger B^\dagger A^\dagger
	\end{align*}}

\section{Μοναδιαίοι Τελεστές}

\definition{\textbf{Τελεστές για τους οποίους ισχύει $U^\dagger = U^{-1}$ ονομάζονται μοναδιαίοι.}}

Οι μοναδιαίοι τελεστές χρησιμοποιούνται για την περιστροφή του συστήματος συντεταγμένων. Επειδή θα πρέπει να ισχύει ότι $(U\psi,U\phi) = (\psi,\phi)$ προκύπτει ο παραπάνω ορισμός $U^\dagger = U^{-1}$.

TODO: Περισσότερη ανάλυση μοναδιαίων αν γίνεται.

\end{document}  