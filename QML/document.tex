\documentclass[11pt, oneside]{article}   	% use "amsart" instead of "article" for AMSLaTeX format
\usepackage{geometry}                		
\usepackage{blindtext}
\usepackage[utf8]{inputenc}
\usepackage{amsmath}
\usepackage[greek,english]{babel}
\usepackage{alphabeta}
\geometry{letterpaper}                   		% ... or a4paper or a5paper or ... 
%\geometry{landscape}                		% Activate for rotated page geometry
%\usepackage[parfill]{parskip}    		% Activate to begin paragraphs with an empty line rather than an indent
\usepackage{graphicx}				% Use pdf, png, jpg, or eps§ with pdflatex; use eps in DVI mode
								% TeX will automatically convert eps --> pdf in pdflatex		
\usepackage{amssymb}
\usepackage{tikz}
\usepackage{quantikz}

%SetFonts

%SetFonts


\title{Εργασία Κβαντικής Μηχανικής Μάθησης}
\author{Αλκίνοος Παπαγεωργόπουλος 1115201900142}
\date{}							% Activate to display a given date or no date

\begin{document}
\maketitle

\centering

\section{Μέρος Α}

Το κύκλωμα που υλοποιείται στην πρώτη περίπτωση εμφανίζεται παρακάτω:

\begin{quantikz}
    \lstick{\text{qubit 0}} & \gate{RX(f_0)} & \gate{Rot(\theta_0, \theta_1, \theta_2)} & \gate{Rot(\theta_6, \theta_7, \theta_8)} & \qw \\
    \lstick{\text{qubit 1}} & \gate{RX(f_1)} & \gate{Rot(\theta_3, \theta_4, \theta_5)} & \gate{Rot(\theta_9, 0, 0)} & \qw \\
\end{quantikz}

Το κύκλωμα χρησιμοποιεί 10 παραμέτρους τις οποίες τις κάνει rotate, συνεπώς χρησιμοποιούμε κωδικοποίηση γωνίας στο κύκλωμα.
Για την εκπαίδευση του κυκλώματος, κάνουμε χρήση του \verb|GradientDescentOptimizer| της Pennylake.

\section{Μέρος Β}

Στα πλαίσια του μέρους Β, εισάγουμε ένα entaglement layer. Το κύκλωμα εμφανίζεται παρακάτω:

\begin{quantikz}
    \lstick{\text{qubit 0}} & \gate{RX(f_0)} & \gate{Rot(\theta_0, \theta_1, \theta_2)} & \ctrl{1} & \gate{Rot(\theta_6, \theta_7, \theta_8)} & \qw \\
    \lstick{\text{qubit 1}} & \gate{RX(f_1)} & \gate{Rot(\theta_3, \theta_4, \theta_5)} & \targ{} & \gate{Rot(\theta_9, \theta_{10}, \theta_{11})} & \qw \\
\end{quantikz}

Το κύκλωμα χρησιμοποιεί 12 παραμέτρους και κάνει χρήση των ήδη υλοποιημένων συναρτήσεων με τη διαφορά ότι έχουμε εισάγει το entaglement layer.

\end{document}  
