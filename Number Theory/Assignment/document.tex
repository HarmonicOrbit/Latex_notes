\documentclass[11pt, oneside]{article}   	% use "amsart" instead of "article" for AMSLaTeX format
\usepackage{geometry}                		
\usepackage{blindtext}
\usepackage[utf8]{inputenc}
\usepackage{amsmath}
\usepackage[greek,english]{babel}
\usepackage{alphabeta}
\geometry{letterpaper}                   		% ... or a4paper or a5paper or ... 
%\geometry{landscape}                		% Activate for rotated page geometry
%\usepackage[parfill]{parskip}    		% Activate to begin paragraphs with an empty line rather than an indent
\usepackage{graphicx}				% Use pdf, png, jpg, or eps§ with pdflatex; use eps in DVI mode
								% TeX will automatically convert eps --> pdf in pdflatex		
\usepackage{amssymb}
\usepackage{xcolor}
%SetFonts

%SetFonts


\title{Εργασία Θεωρία Αριθμών}
\author{\Large Κωνσταντίνος Ζουριδάκης 1115202000254}
\date{}							% Activate to display a given date or no date

\begin{document}
\maketitle

\section{Άσκηση 1}
Θα εφαρμόσουμε τον \textcolor{red}{αλγόριθμο του Ευκλείδη}.

Αρχικά εφαρμόζουμε στους αριθμούς 2200, 1210:

\begin{align*}
	2200 &= 1210 \cdot 1 + 990 \\
	1210 &= 990 \cdot 1 + 220 \\
	990 &= 220 \cdot 4+ 110 \\
	220 &= 110 \cdot 2
\end{align*}
Άρα, $(2200, 1210)= 110$.

Τώρα εφαρμόζουμε τον αλγόριθμο στους αριθμούς 110 και 1078

\begin{align*}
	1078 &= 110 \cdot 9+ 88 \\
	110 &= 88 \cdot 1 + 22 \\
	88 &= 22 \cdot 4
\end{align*}

Οπότε, $(1078, 1210, 2200) = 22$.

Έχουμε:

\begin{align*}
	22 &= 110 - 88 \\
	&= 110 - (1078 - 110 \cdot 9) \\
	&= 10 \cdot 110 - 1078 \\
	&= 10 \cdot (990 - 220 \cdot 4) - 1078 \\
	&= 10 \cdot ((1210-220) - 220 \cdot 4)-1078 \\
	&= 10 \cdot (1210 - 220 \cdot 5) - 1078 \\
	&= - 5 \cdot 2200 + 10 \cdot 1210 - 1078
\end{align*}

Τελικά, $x=-5, y=10, z=-1$

\section{Άσκηση 2}

\subsection{Περίπτωση $n \mod 2 = 0$}

 Αν ο $n$ είναι περιττός, τότε $(-1)^{n+1}=1$.
 Η έκφραση $2^n + (-1)^{n+1} > 3$, οπότε από κριτήριο σύνθετων αριθμών η έκφραση θα έχει ως αποτέλεσμα πάντα σύνθετο αριθμό $\forall n>2$.
 
 \subsection{Περίπτωση $n \mod 2 = 1$}
 
 Αν ο $n$ είναι άρτιος, τότε $(-1)^{n+1}=-1$.
 Η έκφραση πάλι θα είναι $2^n + (-1)^{n+1} > 3$, οπότε από κριτήριο σύνθετων αριθμών η έκφραση θα έχει ως αποτέλεσμα πάντα σύνθετο αριθμό $\forall n>2$.
 
 \section{Άσκηση 3}
 
 \[
 478^{534} + 534^{478} \mod 23
 \]
 
 Το Μικρό Θεώρημα του Φερμάτ μας λέει ότι για έναν πρώτο αριθμό \( p \) και οποιονδήποτε ακέραιο \( a \) που δεν διαιρείται από το \( p \):
 
 \[
 a^{p-1} \equiv 1 \mod p
 \]
 
 Στην περίπτωσή μας \( p = 23 \), οπότε:
 
 \[
 a^{22} \equiv 1 \mod 23
 \]
 
 Υπολογίζουμε τα υπόλοιπα των \( 478 \) και \( 534 \) mod 23.
 
 \[
 478 \div 23 = 20 \ (\text{υπόλοιπο} \ 18), \quad 534 \div 23 = 23 \ (\text{υπόλοιπο} \ 5)
 \]
 
 Άρα:
 
 \[
 478 \equiv 18 \mod 23, \quad 534 \equiv 5 \ (\text{mod} \ 23)
 \]
 
 Τώρα η αρχική έκφραση γίνεται:
 
 \[
 478^{534} + 534^{478} \equiv 18^{534} + 5^{478} \mod 23
 \]
 
 Με το Μικρό Θεώρημα του Φερμά, μπορούμε να μειώσουμε τους εκθέτες mod 22:
 
 \[
 534 \mod 22 = 6, \quad 478 \mod 22 = 18
 \]
 
 Άρα:
 
 \[
 18^{534} + 5^{478} \equiv 18^6 + 5^{18} \mod 23
 \]
 
 \[
 18^2 \equiv 18 \times 18 = 324 \equiv 2 \mod 23
 \]
 \[
 18^4 = (18^2)^2 \equiv 2^2 = 4 \mod 23
 \]
 \[
 18^6 = 18^4 \times 18^2 \equiv 4 \times 2 = 8 \mod 23
 \]
 
 \[
 5^2 = 25 \equiv 2 \mod 23
 \]
 \[
 5^4 = (5^2)^2 = 2^2 = 4 \mod 23
 \]
 \[
 5^8 = (5^4)^2 = 4^2 = 16 \mod 23
 \]
 \[
 5^{16} = (5^8)^2 = 16^2 = 256 \equiv 3 \mod 23
 \]
 \[
 5^{18} = 5^{16} \times 5^2 \equiv 3 \times 2 = 6 \mod 23
 \]
 
 Προσθέτουμε τα αποτελέσματα mod 23:
 
 \[
 18^6 + 5^{18} \equiv 8 + 6 = 14 \mod 23
 \]
 
 Αφού \( 18^6 + 5^{18} \equiv 14 \ (\text{mod} \ 23) \), ο αριθμός \( 478^{534} + 534^{478} \) δεν διαιρείται με το 23.
 
 \section{Άσκηση 4}
 
 \subsection{α'}
 
 \begin{align*}
 	x &\equiv 3 \mod 42 \\
 	x &\equiv 24 \mod 63
 \end{align*}
 
 Έχουμε:
 
 \begin{align*}
 	(42,63) = (21, 42) = (21, 21) = 21 
 \end{align*}
 
 Όμως, το σύστημα γραμμικών ισοδυναμιών έχει μοναδική λύση $\mod 42 \cdot 63 = \mod 2646$, αφού:
 
 \begin{align*}
 	(42,63) | (24-3) = 21 | 21
 \end{align*}
 
 Οπότε, έχουμε:
 
 \begin{align}
 	x \equiv 3 \mod 42 \Rightarrow x=42\lambda + 3
 \end{align}
 
 Συνεπώς:
 
 \begin{align*}
 	x \equiv 24 \mod 63 \stackrel{(1)}{\Rightarrow} 42\lambda + 3 &\equiv 24 \mod 63 \\
 	42\lambda &\equiv 21 \mod 63 \\
 	2\lambda &\equiv 1 \mod 3
 \end{align*}
 
 Όμως, $(2,3) = (1,2) = (1,1) = 1$
 Άρα η ισοδυναμία έχει μοναδική λύση:
 
 \begin{align*}
 	\lambda &\equiv 1 \cdot 2^{\phi(3)-1} \mod 3 \\
 	\lambda &\equiv 1 \cdot 2 \mod 3 \\
 	\lambda &\equiv 2 \mod 3
 \end{align*}
 
 Οπότε καταλήγουμε:
 
 \begin{align}
 	\lambda \equiv 3\mu +2
 \end{align}
 
 \begin{align*}
 	(1) \stackrel{(2)}{\Rightarrow} x=42\cdot(3\mu + 2) + 3 \Rightarrow x=126\mu + 87
 \end{align*}
 
 Άρα
 
 \begin{align*}
 	x \equiv 87 \mod 126
 \end{align*}

\subsection{β'}

\begin{align*}
	4x &\equiv 3 \mod 5 \\
	2x &\equiv 6 \mod 12
\end{align*}

Έχουμε:

\begin{align*}
	(5,12) &= (5,7) = (2,5) = (2,3) = (1,2) = (1,1) = 1 \\
	(4,5) &= (1,4) = (1,1) = 1 \\
	(2,12) &= (2,10) = (2,8) = \dots = (1,1) = 1
\end{align*}

Άρα το σύστημα έχει μία λύση $\mod 60$.

Οπότε έχουμε:

\begin{align}
	4x \equiv 3 \mod 5 \Rightarrow 4x = 5\lambda + 3
\end{align}

Συνεπώς:

\begin{align*}
	2x \equiv 6 \mod 12 \Rightarrow 4x \equiv 12 \mod 24 \stackrel{(1)}{\Rightarrow} 5\lambda +3 \equiv 12 \mod 24 \\
	5\lambda \equiv 9 \mod 24
\end{align*}

Όμως, $(5,24) = (5,19) = (5,14) = (5,9) = (4,5) = (1,4) = (1,1) = 1$. Άρα η ισοδυναμία έχει μοναδική λύση:

\begin{align*}
	\lambda &\equiv 9 \cdot 5^{\phi(24)-1} \mod 24 \\
	\lambda &\equiv 9 \cdot 5^7 \mod 24 \\
	\lambda &\equiv 703125 \mod 24
\end{align*}

Οπότε καταλήγουμε:

\begin{align}
	\lambda = 24\mu + 703125
\end{align}

\begin{align*}
	(3)\stackrel{(4)}{\Rightarrow} 4x = 120\mu + 3515628 \\
	x = 30\mu + 878907 \Rightarrow x \equiv 878907 \mod 30
\end{align*}

\section{Άσκηση 5}

\begin{align*}
	f(x)=2x^4-x^2+6 \equiv 0 \mod 49
\end{align*}

Η πολυωνυμική ισοδυναμία έχει 2 λύσεις οι οποίες είναι:

\begin{align*}
	r &\equiv 1 \mod 7 \\
	r &\equiv 6 \mod 7
\end{align*}

Έχουμε:

\begin{align*}
	f'(x) &= 8x^3 -2x \\
	f'(1) &= 6 \not\equiv 0 \mod 7 \\
	f'(6) &= 1716 \not\equiv 0 \mod 7
\end{align*}

Έχουμε:

\begin{align*}
	f(1) = 7=7w \Rightarrow w=1 \\
	kf'(1)+w \equiv 0 \mod 7 \Rightarrow 6k+1 \equiv 0 \mod 7 \\
	\Rightarrow k \equiv 6 \mod 7
\end{align*}

Άρα η μοναδική λύση που αντιστοιχεί στη λύση $r \equiv 1 \mod 7$ είναι:

\begin{align*}
	y = r + kp = 1 + 6 \cdot7 \equiv 43 \mod 7^2
\end{align*}

Επίσης, έχουμε:

\begin{align*}
	f(6) = 2562 = 7w \Rightarrow w = 366 \\
	kf'(6)+w \equiv 0 \mod 7 \Rightarrow 1716k + 366 \equiv 0 \mod 7 \\
	k \equiv 2 \mod 7
\end{align*}

Άρα η μοναδική λύση που αντιστοιχεί στη λύση $r \equiv 6 \mod 7$ είναι:

\begin{align*}
	y=r+kp = 6 + 366 \cdot 7 \equiv 2568 \mod 7
\end{align*}

\section{Άσκηση 6}

\subsection{α}

\begin{align*}
	x^2 \equiv -270 \mod 773
\end{align*}

Έχουμε:

\begin{align*}
	\left(\frac{-270}{773}\right) = - \left(\frac{2}{773}\right) \left(\frac{3}{773}\right)\left(\frac{3}{773}\right)\left(\frac{3}{773}\right)\left(\frac{5}{773}\right) \\
	\Rightarrow -(-1 \cdot (-1) \cdot (-1) \cdot (-1) \cdot (-1)) = 1
\end{align*}

Άρα η τετραγωνική ισοδυναμία έχει λύση.

\subsection{β}

\begin{align*}
	\left(\frac{88}{175}\right) = \left(\frac{2}{175}\right) \left(\frac{2}{175}\right) \left(\frac{2}{175}\right) \left(\frac{11}{175}\right)\\
	\Rightarrow 1 \cdot 1 \cdot 1 \cdot 1 = 1
\end{align*}

\section{Άσκηση 7}

Πρέπει να βρούμε $p: \left(\frac{-3}{p}\right) = -\left(\frac{3}{p}\right)=1$

Οπότε θα πρέπει:

\begin{align*}
	\left(\frac{3}{p}\right)\left(\frac{-1}{p}\right) = 1
\end{align*}

Αν $p = 1 \mod 4$ από το οποίο συνεπάγεται ότι $\left(\frac{-1}{p}\right)=1$ θα πρέπει $\left(\frac{3}{p}\right)=1$ το οποίο ισχύει για $p = 1 \mod 3$.

Αν $p= 3 \mod 4$ από το οποίο συνεπάγεται ότι $\left(\frac{-1}{p}\right)=-1$, τότε θα πρέπει $\left(\frac{3}{p}\right)=-1$ το οποίο ισχύει για $p=2 \mod 3$.

Οπότε έτσι ψάχνουμε και βρίσκουμε τους αριθμούς που ικανοποιούν αυτές τις δύο συνθήκες.

\section{Άσκηση 8}

Για να βρούμε αν υπάρχουν λύσεις της εξίσωσης:

\begin{align*}
	158 &= 81 + 77 \\
	81 &= 77 +4 \\
	77 &= 4\cdot 19 + 1\\
	4 &= 1\cdot 4
\end{align*}

Οπότε η εξίσωση έχει λύσεις.

Πρέπει να βρούμε $x_0,y_0: 81x_0 + 158y_0 = 1$

Κάνοντας αντίστροφα τις πράξεις του Ευκλείδιου αλγόριθμου και έπειτα από αντικαταστάσεις καταλήγουμε:

\[x_0=-39, y_0 = 20\]

Πολλαπλασιάζοντας με το 5870 για να καταλήξουμε στην αρχική εξίσωση παίρνουμε τις λύσεις:

\[x_0 = -228930, y_0 = 117400\]

\end{document}  
